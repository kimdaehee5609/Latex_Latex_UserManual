\documentclass[12pt,a4paper]{report}
\usepackage{kotex}
\usepackage{setspace}
\usepackage{amssymb,amsfonts,amsmath}


\usepackage{booktabs}

		% ------------------------------ table 
			\usepackage{longtable}			%
			\usepackage[table]{xcolor}
			\usepackage{rotating}
			\usepackage{array}
			\usepackage{tabularx}
			\usepackage{tabulary}
			\usepackage{multirow}

			\usepackage{tabu}				%
			\usepackage{tabto}				%  tabto package  
			\usepackage{subcaption}		%  sub caption package  : 다중 캡션 




			\usepackage{boxedminipage}% 미니 페이지
			\usepackage{framed} % 프레임이 있는 글상자

\begin{document}

		\title{table}
		\maketitle
		
		\doublespace
		
		\newpage
		\tableofcontents
		\listoffigures
		\listoftables
		
		\setlength{\parindent}{0cm}
		
% ========================================== chapter ============
\newpage
\chapter{표 만들기}
		

% -------------------------------------------------------------
\clearpage
\section{Table General rukes}
\null

The typeset of tables should be based on the following rules
\begin{enumerate}
	\item never use vertical lines;
	\item avoid double lines;
	\item place the units in the heading of the table (instead of the body);
	\item do not use quotation marks to repeat the content of cells.
\end{enumerate}



	\begin{table}[hbp]
	\caption{Maximum load and nominal tension}
	\centering
	\begin{tabular}{c l c c c}
	\toprule
	$D$  	&& $P_u$ 	& $\sigma_N$ \\
	(in)		&&(lbs)	& (psi) \\
	\toprule
	5   	& test1   & 285  & 38.00 \\
		& test1   & 285  & 38.00 \\
        	& test1   & 285  & 38.00 \\
	\midrule
	5   	& test1   & 285  & 38.00 \\
		& test1   & 285  & 38.00 \\
		& test1   & 285  & 38.00 \\
	\bottomrule
	\end{tabular}%
	\label{aggiungi}%
	\end{table}%
	
	
	
% -------------------------------------------------------------
\clearpage
\section{Standard Tables}
\null
	
	
	\singlespacing
	\textbf{입력}\\
		\begin{boxedminipage}[t]{1.0\linewidth}
		\small
		%-----------------------------------------		
		\begin{verbatim}
			\begin{tabular}{|l|l|l|}
			\hline
			\multicolumn{3}{|c|}{A Table}\\\hline
			\hline
			1,1 & 1,2 & 1,3 \\\hline
			2,1 & 2,2 & 2,3 \\\cline{1-2}
			3,1 & 3,2 & \\\hline
			\end{tabular}
		\end{verbatim} 
		%-------------------------------------------
		\end{boxedminipage} \\ \\
			
	\textbf{출력}\\
	
		\begin{tabular}{|l|l|l|}
		\hline
		\multicolumn{3}{|c|}{A Table}\\\hline
		\hline
		1,1 & 1,2 & 1,3 \\\hline
		2,1 & 2,2 & 2,3 \\\cline{1-2}
		3,1 & 3,2 & \\\hline
		\end{tabular}

	\doublespacing





% ------------------------------------------------------------- 간격 조정
\clearpage
\section{간격 조정 (Spacing)}
\null
	
	
	% -------------------------------------- page -------------------
	\subsection{Arraystretch}
		{
		\renewcommand{\arraystretch}{1.0}
		\begin{tabular}{|c|l|}
		\hline
		a & Row 1 \\ \hline
		b & Row 2 \\ \hline
		c & Row 2 \\ \hline
		d & Row 4 \\ \hline
		\end{tabular}
		}
		{
		\renewcommand{\arraystretch}{1.2}
		\begin{tabular}{|c|l|}
		\hline
		a & Row 1 \\ \hline
		b & Row 2 \\ \hline
		c & Row 2 \\ \hline
		d & Row 4 \\ \hline
		\end{tabular}
		}
		{
		\renewcommand{\arraystretch}{2.0}
		\begin{tabular}{|c|l|}
		\hline
		a & Row 1 \\ \hline
		b & Row 2 \\ \hline
		c & Row 2 \\ \hline
		d & Row 4 \\ \hline
		\end{tabular}
		}
		
	% -------------------------------------- page -------------------
	\clearpage % 표 때문에 페이지 크리어 시킴
	\subsection{Extrarowheight}
	

		\textbf{입력}
		\singlespacing
			\begin{boxedminipage}[t]{1.0\linewidth}
			%-----------------------------------------		
			\begin{verbatim}
				\usepackage{array}
				...
				{
				\setlength{\extrarowheight}{1.5pt}
				\begin{tabular}{|l|l|}
				\hline
				a & Row 1 \\ \hline
				b & Row 2 \\ \hline
				c & Row 3 \\ \hline
				d & Row 4 \\ \hline
				\end{tabular}
				}
			\end{verbatim} 
			%-------------------------------------------
			\end{boxedminipage} \\ \\
		\doublespacing

	
		{
		\setlength{\extrarowheight}{1.0pt}
		\begin{tabular}{|l|l|}
		\hline
		a & Row 1 \\ \hline
		b & Row 2 \\ \hline
		c & Row 3 \\ \hline
		d & Row 4 \\ \hline
		\end{tabular}
		}
		{
		\setlength{\extrarowheight}{1.5pt}
		\begin{tabular}{|l|l|}
		\hline
		a & Row 1 \\ \hline
		b & Row 2 \\ \hline
		c & Row 3 \\ \hline
		d & Row 4 \\ \hline
		\end{tabular}
		}
		{
		\setlength{\extrarowheight}{2.0pt}
		\begin{tabular}{|l|l|}
		\hline
		a & Row 1 \\ \hline
		b & Row 2 \\ \hline
		c & Row 3 \\ \hline
		d & Row 4 \\ \hline
		\end{tabular}
		}


	% -------------------------------------- page -------------------
	\clearpage % 표 때문에 페이지 크리어 시킴
	\subsection{Bigstruts}
	
	
	
	
	% -------------------------------------- page -------------------
	\clearpage
	\subsection{Column Spacing}
	

			\begin{tabular}{|l|l|}
			\hline
			a & Row 1 \\ \hline
			b & Row 2 \\ \hline
			c & Row 3 \\ \hline
			\end{tabular}\\

			\setlength{\tabcolsep}{6pt}
			\begin{tabular}{|l|l|}
			\hline
			a & Row 1 \\ \hline
			b & Row 2 \\ \hline
			c & Row 3 \\ \hline
			\end{tabular}			tabcolsep = 6pt
		
			\setlength{\tabcolsep}{0pt}
			\begin{tabular}{|l|l|}
			\hline
			a & Row 1 \\ \hline
			b & Row 2 \\ \hline
			c & Row 3 \\ \hline
			\end{tabular}			tabcolsep = 0pt
	
			\setlength{\tabcolsep}{10pt}
			\begin{tabular}{|l|l|}
			\hline
			a & Row 1 \\ \hline
			b & Row 2 \\ \hline
			c & Row 3 \\ \hline
			\end{tabular}			tabcolsep = 10pt
			
			\setlength{\tabcolsep}{20pt}
			\begin{tabular}{|l|l|}
			\hline
			a & Row 1 \\ \hline
			b & Row 2 \\ \hline
			c & Row 3 \\ \hline
			\end{tabular}			tabcolsep = 20pt
			
			\setlength{\tabcolsep}{30pt}
			\begin{tabular}{|l|l|}
			\hline
			a & Row 1 \\ \hline
			b & Row 2 \\ \hline
			c & Row 3 \\ \hline
			\end{tabular}			tabcolsep = 30pt
			
			\setlength{\tabcolsep}{1.0ex}
			\begin{tabular}{|l|l|}
			\hline
			a & Row 1 \\ \hline
			b & Row 2 \\ \hline
			c & Row 3 \\ \hline
			\end{tabular}			tabcolsep = 1.0ex
			\setlength{\tabcolsep}{6pt}
	
	
	
	
% -------------------------------------------------------------
\newpage  
\section{Decimal Point Alignment}
\null


\begin{tabular}{|l|r@{.}l|l|}
a & \multicolumn{2}{c|}{b} & c \\
test & 2 & 8 & test \\
test & 1 & 45 & test \\
test & 0 & 5 & test
\end{tabular}


% -------------------------------------------------------------
\newpage  
\section{Vertical Alignment and Text Wrapping}
\null


			p\{width\} Top align, the same as usual.\\
			m\{width\} Middle align\\
			b\{width\} Bottom align]]
























% -------------------------------------------------------------
\newpage  
\section{Table}
\null

	\textbf{입력}
		\begin{minipage}[t]{0.5\textwidth}
		\singlespacing
		\begin{verbatim}
				\begin{table}[h]
					\caption{Nonlinear Model Results}
					\centering
					\begin{tabular}{c c c c c}
					\toprule
					Case & Method \#1 & Method\#2 & Method\#3 \\ [0.5ex]
					\midrule
					1   & 50   & 837  & 970 \\
					2   & 47   & 877  & 230 \\
					3   & 31   & 25   & 415  \\
					4   & 35   & 144  & 2356 \\
					5   & 45   & 300  & 556  \\ [1ex]
					\bottomrule
					\end{tabular}%
					\label{table:nonlin}%
				\end{table}
		\end{verbatim}
		\end{minipage}\\

	\textbf{출력}
				\begin{table}[ht]
					\caption{Nonlinear Model Results}
					\centering
					\begin{tabular}{c c c c c}
					\toprule
					Case & Method \#1 & Method\#2 & Method\#3 \\ [0.5ex]
					\midrule
					1   & 50   & 837  & 970 \\
					2   & 47   & 877  & 230 \\
				    3   & 31   & 25   & 415  \\
				    4   & 35   & 144  & 2356 \\
				    5   & 45   & 300  & 556  \\ [1ex]
				    \bottomrule
				    \end{tabular}%
				  \label{table:nonlin}%
				\end{table}
				
		표 \ref{table:nonlin}
		

% -------------------------------------------------------------
%
%		Multi row cells
%
% -------------------------------------------------------------
\newpage  
\section{Multi row cells}
\null
	\singlespacing
	\textbf{입력}\\
		\begin{boxedminipage}[t]{1.0\linewidth}
		\small
		%-----------------------------------------		
		\begin{verbatim}
			\begin{table}[ht]
				\caption{Multrow cells}
				\centering
			    \begin{tabular}{c l c c}
			    \toprule
				\multicolumn{2}{c}{$D$} 	& $P_u$ 	& $\sigma_N$ \\
				\multicolumn{2}{c}{(in)} 	& (lbs) 	& (psi) \\
			    \toprule
			    \multirow{3}*{5}  	& test1   & 285  & 38.00 \\
							    	& test1   & 285  & 38.00 \\
							    	& test1   & 285  & 38.00 \\
			    \midrule
			    \multirow{3}*{10}	& test1   & 285  & 38.00 \\
								& test1   & 285  & 38.00 \\
								& test1   & 285  & 38.00 \\
			    \bottomrule
			    \end{tabular}
			  \label{table:nonlin}
			\end{table}
		\end{verbatim} 
		%-------------------------------------------
		\end{boxedminipage} \\ \\
		
	\textbf{출력}\\
	\vspace{-2.0em}
			\begin{table}[h]
				\caption{Multrow cells}
				\centering
			    \begin{tabular}{c l c c}
			    \toprule
				\multicolumn{2}{c}{$D$} 	& $P_u$ 	& $\sigma_N$ \\
				\multicolumn{2}{c}{(in)} 	& (lbs) 	& (psi) \\
			    \toprule
			    \multirow{3}*{5}  	& test1   & 285  & 38.00 \\
							    	& test1   & 285  & 38.00 \\
							    	& test1   & 285  & 38.00 \\
			    \midrule
			    \multirow{3}*{10}	& test1   & 285  & 38.00 \\
								& test1   & 285  & 38.00 \\
								& test1   & 285  & 38.00 \\
			    \bottomrule
			    \end{tabular}%
			  \label{table:nonlin}%
			\end{table}%
	\doublespacing


% ------------------------------------------------------------- 1
\clearpage
\section{Table Example-1}
\null

	\singlespacing
	\textbf{입력}\\
		\begin{boxedminipage}[t]{1.0\linewidth}
		\small
		%-----------------------------------------		
		\begin{verbatim}	
			\begin{table}[h]
				\caption{Preformance}
				\centering
				\begin{tabular}{c rrrrrrr }
				\hline \hline
				Audi name & \multicolumn{7}{c}{ sum of }  \\ [0.5ex]
				\hline
				police   	& 5 & -1 & 5  & 5 & -7 & -5 & 3 \\
				midnight 	& 5 & -1 & 5  & 5 & -7 & -5 & 3 \\
				news   	& 5 & -1 & 5  & 5 & -7 & -5 & 3 \\
				\hline
				\end{tabular}%
				\label{table:hresult}%
				\end{table}%
		\end{verbatim} 
		%-------------------------------------------
		\end{boxedminipage} \\ \\
		
		\textbf{출력}\\
		\vspace{-2.0em}
				\begin{table}[h]
					\caption{Preformance}
					\centering
				    \begin{tabular}{c rrrrrrr }
				    \hline \hline
					Audi name & \multicolumn{7}{c}{ sum of }  \\ [0.5ex]
				    \hline
				    police   	& 5 & -1 & 5  & 5 & -7 & -5 & 3 \\
				    midnight 	& 5 & -1 & 5  & 5 & -7 & -5 & 3 \\
				    news   	& 5 & -1 & 5  & 5 & -7 & -5 & 3 \\
				    \hline
				    \end{tabular}%
				  \label{table:hresult}%
				\end{table}%
				
	표 \ref{table:hresult}
	\doublespacing
		

% ------------------------------------------------------------- 2
\newpage 
\section{Table Example-2}
\null

	\singlespacing
	\textbf{입력}\\
		\begin{boxedminipage}[t]{1.0\linewidth}
		\small
		%-----------------------------------------		
		\begin{verbatim}	
			\begin{table}[h]
				\caption{Performance After Post Filtering}
				\centering
				\begin{tabular}{ l c c rrrrrrr }
				\hline \hline
				Audi & Audibility & Decision & \multicolumn{7}{c}{sum of}\\ [0.5ex]
				\hline
				& 	& soft 	&  5 & -1 & 5  & 5 & -7 & -5 & 3 \\[-1.0ex]
				\raisebox{1.5ex} {police} & \raisebox{1.5ex} {5} 
					& hard	&  5 & -1 & 5  & 5 & -7 & -5 & 3 \\[1.0ex]
				& 	& soft 	&  5 & -1 & 5  & 5 & -7 & -5 & 3 \\[-1.0ex]
				\raisebox{4.0ex} {police} & \raisebox{2.0ex} {5} 
					& hard	&  5 & -1 & 5  & 5 & -7 & -5 & 3 \\[1.0ex]
				& 	& soft 	&  5 & -1 & 5  & 5 & -7 & -5 & 3 \\[-1.0ex]
				\raisebox{0.0ex} {police} & \raisebox{2.0ex} {5} 
					& hard	&  5 & -1 	& 5  & 5 & -7 & -5 & 3 \\[0.0ex]
				\hline
				\end{tabular}%
				\label{tab:PPer}%
			\end{table}%
		\end{verbatim} 
		%-------------------------------------------
		\end{boxedminipage} \\ \\

		\textbf{출력}\\
		\vspace{-2.0em}
		\begin{table}[h]
			\caption{Performance After Post Filtering}
			\centering
			\begin{tabular}{ l c c rrrrrrr }
			\hline \hline
			Audi & Audibility & Decision & \multicolumn{7}{c}{ sum of }  \\ [0.5ex]
			\hline
			& 	& soft 	&  5 & -1 & 5  & 5 & -7 & -5 & 3 \\[-1.0ex]
			\raisebox{1.5ex} {police} & \raisebox{1.5ex} {5} 
				& hard	&  5 & -1 & 5  & 5 & -7 & -5 & 3 \\[1.0ex]
			& 	& soft 	&  5 & -1 & 5  & 5 & -7 & -5 & 3 \\[-1.0ex]
			\raisebox{4.0ex} {police} & \raisebox{2.0ex} {5} 
				& hard	&  5 & -1 & 5  & 5 & -7 & -5 & 3 \\[1.0ex]
			& 	& soft 	&  5 & -1 & 5  & 5 & -7 & -5 & 3 \\[-1.0ex]
			\raisebox{0.0ex} {police} & \raisebox{2.0ex} {5} 
				& hard	&  5 & -1 	& 5  & 5 & -7 & -5 & 3 \\[0.0ex]
			\hline
			\end{tabular}%
			\label{tab:PPer}%
		\end{table}%


표 \ref{tab:PPer}


% ------------------------------------------------------------- 3
\newpage
\section{Table Example - sidewaystable}

	\singlespacing
	\textbf{입력}\\
		\begin{boxedminipage}[t]{1.0\linewidth}
		\small
		%-----------------------------------------		
		\begin{verbatim}	
			\begin{sidewaystable} [h]
				\caption{Performance After Post Filtering}
				\centering
				\begin{tabular}{ l c c rrrrrrr }
					\hline \hline
					Audi & Audibility & Decision & \multicolumn{7}{c}{sum of}\\[0.5ex]
					\hline
					& 	& soft 	&5 & -1 & 5  & 5 & -7 & -5 & 3 \\[-1.0ex]
					\raisebox{1.5ex} {police} & \raisebox{1.5ex} {5} 
						& hard   &5 & -1 & 5  & 5 & -7 & -5 & 3 \\[1.0ex]
					& 	& soft 	&5 & -1 & 5  & 5 & -7 & -5 & 3 \\[-1.0ex]
					\raisebox{4.0ex} {police} & \raisebox{2.0ex} {5} 
						& hard	&5 & -1 & 5  & 5 & -7 & -5 & 3 \\[1.0ex]
					& & soft &  5 & -1 & 5  & 5 & -7 & -5 & 3 \\[-1.0ex]
					\raisebox{0.0ex} {police} & \raisebox{2.0ex} {5} 
						& hard	&5 & -1 & 5  & 5 & -7 & -5 & 3 \\[0.0ex]
					\hline
				\end{tabular}%
				\label{tab:LPer}%
			\end{sidewaystable}%
		\end{verbatim} 
		%-------------------------------------------
		\end{boxedminipage} \\ \\


		항상 짝수 쪽으로 가서 배치되는 것 같다.
	표 \ref{tab:LPer}
	
	\doublespacing
	\newpage
			\begin{sidewaystable} [h]
				\caption{Performance After Post Filtering}
				\centering
				\begin{tabular}{ l c c rrrrrrr }
					\hline \hline
					Audi & Audibility & Decision & \multicolumn{7}{c}{sum of}\\[0.5ex]
					\hline
					& 	& soft 	&5 & -1 & 5  & 5 & -7 & -5 & 3 \\[-1.0ex]
					\raisebox{1.5ex} {police} & \raisebox{1.5ex} {5} 
						& hard   &5 & -1 & 5  & 5 & -7 & -5 & 3 \\[1.0ex]
					& 	& soft 	&5 & -1 & 5  & 5 & -7 & -5 & 3 \\[-1.0ex]
					\raisebox{4.0ex} {police} & \raisebox{2.0ex} {5} 
						& hard	&5 & -1 & 5  & 5 & -7 & -5 & 3 \\[1.0ex]
					& & soft &  5 & -1 & 5  & 5 & -7 & -5 & 3 \\[-1.0ex]
					\raisebox{0.0ex} {police} & \raisebox{2.0ex} {5} 
						& hard	&5 & -1 & 5  & 5 & -7 & -5 & 3 \\[0.0ex]
					\hline
				\end{tabular}%
				\label{tab:LPer}%
			\end{sidewaystable}%





% ------------------------------------------------------------- 4
\clearpage
\section{Table Example - tabular만 사용한 경우}

	\singlespacing
	\textbf{입력}\\
		\begin{boxedminipage}[t]{1.0\linewidth}
		\small
		%-----------------------------------------		
		\begin{verbatim}	
			\begin{tabular}{lllll}
				one & two & three & four & five \\
				six & seven & eight & nine & ten \\
				eleven & twelve & thirteen \\
			\end{tabular} 
		\end{verbatim} 
		%-------------------------------------------
		\end{boxedminipage} \\ \\

		\textbf{출력}
		\begin{tabular}{lllll}
			\hline
			one & two & three & four & five \\
			six & seven & eight & nine & ten \\
			eleven & twelve & thirteen \\
			\hline
		\end{tabular} \\
		
		\textbf{출력} \\
		\begin{tabular}{lllll}
			\toprule
			one & two & three & four & five \\
			six & seven & eight & nine & ten \\
			eleven & twelve & thirteen \\
			\bottomrule
		\end{tabular} \\
		
	\doublespacing
		\textbf{출력} \\
		\begin{tabular}{lllll}
			\toprule
			one & two & three & four & five \\
			six & seven & eight & nine & ten \\
			eleven & twelve & thirteen \\
			\bottomrule
		\end{tabular} \\

% ------------------------------------------------------------- 4
\clearpage
\section{Table Example-4-1}

	\singlespacing
	\textbf{입력}\\
		\begin{boxedminipage}[t]{1.0\linewidth}
		\small
		%-----------------------------------------		
		\begin{verbatim}	
			\begin{tabular}{lllll}
				one & two & three & four & five \\
				six & seven & eight & nine & ten \\
				eleven & twelve & thirteen \\
			\end{tabular} 
		\end{verbatim} 
		%-------------------------------------------
		\end{boxedminipage} \\

		\textbf{출력}\\
		
		\vspace{-2.0em}
		\begin{table}[h]
			\centering 
			\caption{Performance After Post Filtering}
			\begin{tabular}{l l l l l}
				\hline
				one & two & three & four & five \\
				six & seven & eight & nine & ten \\
				eleven & twelve & thirteen \\
				\hline
			\end{tabular} \\
		\end{table}
		
	\doublespacing
		
	표야 날라가지 마라 	\\
	newpage하면 날라가버리고\\
	clearpage하면 안 날라간다.\\
	
	
% ------------------------------------------------------------- 4
\clearpage
\section{Table Example - list }
	
\begin{list}{-}{body code}
\item first item
\item second item
\item third item
\end{list}





% -------------------------------------------------------------
\newpage
\section{table의 작성}

		\begin{table}[htbp]
		\centering
		\caption{표 7-16 유출 노즈부의 최소 평면곡선 반지름 계산}
		\begin{tabular}{p{2cm}p{2cm}p{2cm}p{2cm}p{2cm}}
			\hline
			본선 설계속도 (km/hr) & 노즈
			통과속도 & 노즈부의 평면곡선 반지름 계산값 & 노즈부의 최소 평면곡선반지름 & 감속도 \\
			\hline
			120   & 60    & 236   & 250   & 1.0  \\
			110   & 58    & 220   & 230   & 1.0  \\
			100   & 55    & 198   & 200   & 1.0  \\
			90    & 53    & 184   & 185   & 1.0  \\
			80    & 50    & 164   & 170   & 1.0  \\
			70    & 45    & 132   & 140   & 1.0  \\
			60    & 40    & 105   & 110   & 1.0  \\
		\hline
		\end{tabular}%
		\label{tab:addlabel}%
		\end{table}%


		\begin{table}[htbp]
		\centering
		\caption{표 7-16 유출 노즈부의 최소 평면곡선 반지름 계산} 
		\begin{tabular}{p{2cm}p{2cm}p{2cm}p{2cm}p{2cm}}
		\hline
		\begin{minipage}[t]{2cm}본선\\설계속도\\(km/hr)\end{minipage}& 
		\begin{minipage}[t]{2cm}노즈\\통과\\속도\\(km/hr)\end{minipage}& 
		\begin{minipage}[t]{2cm}노즈부의\\평면곡선\\반지름\\계산값\end{minipage}& 
		\begin{minipage}[t]{2cm}노즈부의\\최소\\평면곡선\\반지름\end{minipage}& 
		감속도 \\
			\hline
			120   & 60    & 236   & 250   & 1.0  \\
			110   & 58    & 220   & 230   & 1.0  \\
			100   & 55    & 198   & 200   & 1.0  \\
			90    & 53    & 184   & 185   & 1.0  \\
			80    & 50    & 164   & 170   & 1.0  \\
			70    & 45    & 132   & 140   & 1.0  \\
			60    & 40    & 105   & 110   & 1.0  \\
		\hline
		\end{tabular}%
		\label{tab:addlabel}%
		\end{table}%

% -------------------------------------------------------------
\newpage
\section{table안에 수식 작성}
\null

	\singlespacing
	\textbf{입력}\\
		\begin{boxedminipage}[t]{1.0\linewidth}
		\small
		%-----------------------------------------		
		\begin{verbatim}	
		\begin{tabular}{|c|c|}
		\hline
		Cylindrical & 
		$\displaystyle{ 	{1 \over \rho}
						{\partial \over \partial\rho}
						\left(\rho {\partial f \over \partial \rho}\right)
						+ 	{1 \over \rho^2}
							{\partial^2 f \over \partial \phi^2}  
						+ 	{\partial^2 f \over \partial z^2}
						}$\\
		\hline
		Spherical & 
		$\displaystyle{ 	{1 \over r^2}
						{\partial \over \partial r}\!
						\left(r^2 {\partial f \over \partial r}\right)
						\!+\!
						{1 \over r^2\!\sin\theta}
						{\partial \over \partial \theta}
						\!
						\left(\sin\theta {\partial f \over \partial \theta}\right)
						\!+\!
						{1 \over r^2\!\sin^2\theta}
						{\partial^2 f \over \partial \phi^2}
						}$\\
		\hline
		\end{tabular}
		\end{verbatim} 
		%-------------------------------------------
		\end{boxedminipage} \\

		\textbf{출력}\\
		
	\doublespacing

		\begin{tabular}{|c|c|}
		\hline
		Cylindrical & 
		$\displaystyle{ 	{1 \over \rho}
						{\partial \over \partial\rho}
						\left(\rho {\partial f \over \partial \rho}\right)
						+ 	{1 \over \rho^2}
							{\partial^2 f \over \partial \phi^2}  
						+ 	{\partial^2 f \over \partial z^2}
						}$\\
		\hline
		Spherical & 
		$\displaystyle{ 	{1 \over r^2}
						{\partial \over \partial r}\!
						\left(r^2 {\partial f \over \partial r}\right)
						\!+\!
						{1 \over r^2\!\sin\theta}
						{\partial \over \partial \theta}
						\!
						\left(\sin\theta {\partial f \over \partial \theta}\right)
						\!+\!
						{1 \over r^2\!\sin^2\theta}
						{\partial^2 f \over \partial \phi^2}
						}$\\
		\hline
		\end{tabular}
		
% -------------------------------------------------------------
\newpage
\section{table안에 수식 작성 - raisebox 명령 사용 }
\null
		
		
		\begin{tabular}{|c|c|}
		\hline
		\raisebox{28pt}{Cylindrical }& 
		$\displaystyle{ 	{1 \over \rho}
						{\partial \over \partial\rho}
						\left(\rho {\partial f \over \partial \rho}\right)
						+ 	{1 \over \rho^2}
							{\partial^2 f \over \partial \phi^2}  
						+ 	{\partial^2 f \over \partial z^2}
						}$\\
		\hline
		\raisebox{28pt}{Spherical} & 
		$\displaystyle{ 	{1 \over r^2}
						{\partial \over \partial r}\!
						\left(r^2 {\partial f \over \partial r}\right)
						\!+\!
						{1 \over r^2\!\sin\theta}
						{\partial \over \partial \theta}
						\!
						\left(\sin\theta {\partial f \over \partial \theta}\right)
						\!+\!
						{1 \over r^2\!\sin^2\theta}
						{\partial^2 f \over \partial \phi^2}
						}$\\
		\hline
		\end{tabular}
		
		
		
		
% ========================================== chapter ============
\newpage
\chapter{Long Table}


	% -------------------------------------------------------------
	\newpage
	\section{	Long Table}





% ========================================== chapter ============
\newpage
\chapter{Tabularx}


		
	% -------------------------------------------------------------
	\newpage
	\section{	Tabularx}


		\begin{verbatim}	
			\begin{tabularx} {<width>}{,preamble>}
			\end{tabularx} 
		\end{verbatim} 





% 	========================================== chapter ===============================================
	\newpage
	\chapter{Tabu}




%	---------------------------------------------------------------------------------------------------
%
%
%	
%	---------------------------------------------------------------------------------------------------
	\clearpage
	\section{Table의 크기 설정}



%	---------------------------------------------------------------------------------------------------
%
%
%	
%	---------------------------------------------------------------------------------------------------
	\clearpage
	\section{tabu 줄 간격 조정}

		\begin{verbatim}
		\tabulinesep=2ex
		\end{verbatim}




%	---------------------------------------------------------------------------------------------------
%
%
%	
%	---------------------------------------------------------------------------------------------------
	\clearpage
	\section{Table 에서 괘선 긋기}


	\paragraph{tabucline}

		\begin{table}[h]
		\caption{접합부의 항목모드}
		\tabulinesep=2ex
		\begin{tabu} to 1.0\textwidth { X[r,m, 1.0] X[c,m, 1.0] }
		\tabucline[0.2ex]{-}		
		구분			&줄 두께\\
		\tabucline[0.1ex]{-}		
		제목 상단줄 	&0.20 ex \\
		\tabucline[0.01ex]{-}		
		제목 하단줄 	&0.10 ex \\
		\tabucline[0.01ex]{-}		
		본문 구분줄 	&0.01 ex \\
		\tabucline[0.01ex]{-}		
		본문 마감줄 	&0.10 ex \\
		\tabucline[0.1ex]{-}		
		\end{tabu}
		\end{table}


		
		



%	---------------------------------------------------------------------------------------------------
%
%
%	
%	---------------------------------------------------------------------------------------------------
	\clearpage
	\section{표의 caption과 본문의 간격 조정}

		\begin{verbatim}
		\setlength{\abovecaptionskip}{0em}
		\setlength{\belowcaptionskip}{1em}

		usepackage caption 
		\captionsetup[table]{aboveskip=0pt}
		\captionsetup[table]{belowskip=10pt}
		\captionsetup[table]{font=small,skip=0pt}
		\captionsetup[figure]{font=small,skip=0pt}
		\end{verbatim}


		\begin{table}[h]
		\setlength{\abovecaptionskip}{0em}
		\setlength{\belowcaptionskip}{1em}

		\caption{접합부의 항목모드}
%		\tabulinesep=2ex
		\begin{tabu} to 1.0\textwidth { X[r,m, 1.0] X[c, 1.0] X[3.0] }
		\tabucline[0.2ex]{-}		
		항복모드	&형상		&내용\\
		\tabucline[0.2ex]{-}		0
		$I_m$		&\\
		\tabucline[0.01ex]{-}		
		$I_s$			&\\
		\tabucline[0.1ex]{-}		
		\end{tabu}
		\end{table}

		\begin{table}[h]
		\caption{접합부의 항목모드}
%		\tabulinesep=2ex
		\begin{tabu} to 1.0\textwidth { X[r,m, 1.0] X[c, 1.0] X[3.0] }
		\tabucline[0.2ex]{-}		
		항복모드	&형상		&내용\\
		\tabucline[0.2ex]{-}		
		$I_m$		&\\
		\tabucline[0.01ex]{-}		
		$I_s$			&\\
		\tabucline[0.1ex]{-}		
		\end{tabu}
		\end{table}



%	---------------------------------------------------------------------------------------------------
%
%
%	
%	---------------------------------------------------------------------------------------------------
	\clearpage
	\section{두개의 표에 각각 caption을 달고 전체 캡션을 다는 경우}

			
		\begin{verbatim}
		
				\begin{table}[h]
				\caption{접합부의 항목모드}
		
				\begin{subtable}[h]{\textwidth}
				\begin{center}
				\caption{접합부의 항목모드}
				\begin{tabu} to 0.8\textwidth { X[r,m, 1.0] X[c, 1.0] X[3.0] }
				\tabucline[0.2ex]{-}		
				항복모드	&형상		&내용\\
				\tabucline[0.2ex]{-}		
				$I_m$		&\\
				\tabucline[0.01ex]{-}		
				$I_s$			&\\
				\tabucline[0.1ex]{-}		
				\end{tabu}
				\end{center}
				\end{subtable}
		
				\begin{subtable}[h]{\textwidth}
				\caption{접합부의 항목모드}
				\begin{center}
				\begin{tabu} to 0.80\textwidth { X[r,m, 1.0] X[c, 1.0] X[3.0] }
				\tabucline[0.2ex]{-}		
				항복모드	&형상		&내용\\
				\tabucline[0.2ex]{-}		
				$I_m$		&\\
				\tabucline[0.01ex]{-}		
				$I_s$			&\\
				\tabucline[0.1ex]{-}		
				\end{tabu}
				\end{center}
				\end{subtable}
		
				\end{table}
		\end{verbatim}


		\begin{table}[h]
		\caption{접합부의 항목모드}

		\begin{subtable}[h]{\textwidth}
		\begin{center}
		\caption{접합부의 항목모드}
		\begin{tabu} to 0.8\textwidth { X[r,m, 1.0] X[c, 1.0] X[3.0] }
		\tabucline[0.2ex]{-}		
		항복모드	&형상		&내용\\
		\tabucline[0.2ex]{-}		
		$I_m$		&\\
		\tabucline[0.01ex]{-}		
		$I_s$			&\\
		\tabucline[0.1ex]{-}		
		\end{tabu}
		\end{center}
		\end{subtable}

		\begin{subtable}[h]{\textwidth}
		\caption{접합부의 항목모드}
		\begin{center}
		\begin{tabu} to 0.80\textwidth { X[r,m, 1.0] X[c, 1.0] X[3.0] }
		\tabucline[0.2ex]{-}		
		항복모드	&형상		&내용\\
		\tabucline[0.2ex]{-}		
		$I_m$		&\\
		\tabucline[0.01ex]{-}		
		$I_s$			&\\
		\tabucline[0.1ex]{-}		
		\end{tabu}
		\end{center}
		\end{subtable}

		\end{table}








%	---------------------------------------------------------------------------------------------------
%
%
%	
%	---------------------------------------------------------------------------------------------------
	\clearpage
	\section{여러개의 표를 사용하여 하나의 표작성}


		\begin{table}[h]
		\caption{접합부의 항목모드}
		\tabulinesep=2ex

		\tabulinesep=2ex
		\begin{tabu} to 1.0\textwidth { | X[c,m, 1.0] | X[c, 1.0] | X[c,1.0] | }
		\tabucline[0.2ex]{-}		
		항복모드	&형상		&내용\\
		\end{tabu} \\[-0.4ex]

		\tabulinesep=1ex

		\begin{tabu} to 1.0\textwidth { | X[r,m, 1.0] | X[c, 1.0] | X[1.0] | }
		\tabucline[0.2ex]{-}		
		항복모드	&형상		&내용\\
		\tabucline[0.2ex]{-}		
		$I_m$		&	&\\
		\tabucline[0.01ex]{-}		
		$I_s$		&	&\\
		\end{tabu} \\[-0.4ex]

		\begin{tabu} to 1.0\textwidth { | X[r,m, 3.0] | X[c, 2.0] | X[1.0] | }
		\tabucline[0.2ex]{-}		
		항복모드	&형상		&내용\\
		\tabucline[0.1ex]{-}		
		$I_m$			& 		&\\
		\tabucline[0.01ex]{-}		
		$I_s$			&		&\\
		\end{tabu} \\[-0.4ex]

		\begin{tabu} to 1.0\textwidth { | X[r,m, 1.0] | X[c, 1.0] | X[3.0] | }
		\tabucline[0.2ex]{-}		
		항복모드	&형상		&내용\\
		\tabucline[0.1ex]{-}		
		$I_m$		&	&\\
		\tabucline[0.01ex]{-}		
		$I_s$		&	&\\
		\tabucline[0.1ex]{-}		
		\end{tabu} \\[-0.4ex]


		\end{table}


%	---------------------------------------------------------------------------------------------------
%
%
%	
%	---------------------------------------------------------------------------------------------------
	\clearpage
	\section{long tabu}

				longtable package 사용


		%		\tabulinesep=0.6ex
				\begin{longtabu} to 1.0\textwidth { X[r,m, 1.2] X[l,m, 1.6] X[r, 0.8] X[c,m, 1.8] }
				\tabucline[0.2ex]{-}		
				품명				&규격		&단가	&비고\\
				\tabucline[0.1ex]{-}		
				\endfirsthead
				\tabucline[0.2ex]{-}		
				품명				&규격		&단가	&비고\\
				\tabucline[0.1ex]{-}		
				\endhead
				\tabucline[0.1ex]{-}		
				\endfoot
				\tabucline[0.1ex]{-}		
				\endlastfoot
				쌀				& 5 kg		& 21,000	&\\
				찹쌀				&			&		&\\
				현미				&			&		&\\
				누렁지			&			&16,100	&\\
				볶음밥 			&			&		&\\
				즉석밥 			&			&		&\\
				떡국	 			&			&		&\\
				\tabucline[0.01ex]{-}		
				김치				&				&		&\\
				단무지			&				&		&\\
				\tabucline[0.01ex]{-}		
				감자				&				&		&\\
				고구마			&				&		&\\
				\tabucline[0.1ex]{-}		
				\end{longtabu} 


 
\end{document}


