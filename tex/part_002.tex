


%	---------------------------------------------------------------------------------------------------
%
%	\part{글자 모양}
%
%	---------------------------------------------------------------------------------------------------







% ========================================== chapter ============
\newpage  
\chapter{글꼴 크기}

	% -------------------------------------- page -------------------
		\minitoc				% Creating an actual minitoc

		
	% --------------- subsection -------------- 글꼴 크기 조정
	\clearpage
	\section{글꼴 크기 조정}
	\null
	
	
		위드프로세서에서는 폰트의 크기를 조정함으로써 글자 크기를 조정했지만 \LaTeX에서는
		문서의 기본 글꼴 크기가 정해져 있고, 이에 비례하여 글자 크기를 조정한다.
		
		
			\begin{table}[hbp]
			\caption{글자 크기}
			\centering
			\begin{tabular}{l  l  l }
			\toprule
			명령어 & 결과 \\
			\toprule
			\verb|\tiny|				& \tiny결과 &\normalsize{결과}\\
			\verb|\script size|		& \scriptsize{결과} \\
			\verb|\foot notee size| 	& \footnotesize{결과} \\
			\verb|\small|			& \small{결과} \\
			\verb|\normal size| 		& \normalsize{결과} \\
			\verb|\large| 			& \large{결과} \\
			\verb|\Large| 			& \Large{결과} \\
			\verb|\LARGE| 			& \LARGE{결과} \\
			\verb|\huge| 			& \huge{결과} \\
			\verb|\Huge| 			& \Huge{결과} \\
			\bottomrule
			\end{tabular}%
			\end{table}%
			
			
			
			
		
				
	% --------------- subsection --------------
	\clearpage
	\section{큰 글꼴 사용 : ext size package }
	\null
	
		\subsection*{큰 글꼴 사용}
	
			표준 크기 (10,11pt)가 충분하지 않다면, extsizes 패키지를 이용하면 된다.\\
			
			\begin{boxedminipage}[c]{1.0\linewidth}
			\begin{verbatim}
				\documentclass[17pt]{ext article}
			\end{verbatim}
			\end{boxedminipage}
			
			문서의 기본 글꼴 크기를 바꾸려면 위 명령어를 프리엠블에 쓴다. 
			\\\\
			이 패키지는 확장된 표준 문서 클래스 옵션을 제공하여 8,9,10,11,12,14,17,20 포인트 문서를 
			작성할 수있도록 해준다.
		
	% --------------- subsection --------------
	\clearpage
	\section{큰 글꼴 사용 : fix-cm package }
	\null
	
			LaTeX은 글자 크기를 tiny, scriptsize, footnotesize, small, normalsize, 
			large, Large, LARGE, huge, Huge 명령을 이용해서 정합니다.
			nomalsize의 크기가 정해지면 그에 맞게 다른 크기가 변합니다.
			그런데, 가장 큰 Huge라 해도 그리 크지가 않습니다. 
			만약 문서의 표지 제목 등 큰 글자가 필요할 때는 어떻게 할까요? 
			여러가지 방법이 있지만 fontsize 명령을 사용하는 게 제일 편합니다.\\
			
			\singlespace
			\begin{boxedminipage}[c]{1.0\linewidth}
				\begin{verbatim}
					\usepackage{fix-cm}
					...
					\fontsize{<font size>}{<baselineskip>}
				\end{verbatim}
			\end{boxedminipage}\\

			간단한 예는 다음과 같습니다.\\
			
			\begin{boxedminipage}[c]{1.0\linewidth}
				\begin{verbatim}
					\documentclass[a4paper]{article}
					\usepackage{kotex}
					\usepackage{fix-cm}
					\begin{document}
					\fontsize{80}{70} \rmfamily 큰 글자 A
					\end{document}
				\end{verbatim}
			\end{boxedminipage}\\
			\doublespace
			
			빨간색의 코드를 넣어야 영어 글자가 제대로 커집니다. 결과는 다음과 같습니다.		\\
			{\Huge Huge size 큰 글자 A}\\
			{\fontsize{70}{70} \rmfamily 70 size 큰 글자 A}
					
			\newpage
			{\fontsize{20}{20}  \rmfamily 20 point \par}
			큰 글자 A \par
			{\fontsize{30}{35} \rmfamily 큰 글자 A \par}
			큰 글자 A \par
			{\fontsize{30}{35} \selectfont 큰 글자 A Big\\ More \par}	
			큰 글자 A \par
			\doublespace
																														\subsection*{사용시 주의 사항}
																														반드시\{ \} 사용하여 범위를 제한 할것
																																
																																		
																																				
% ========================================== chapter ============
\newpage  
\chapter{글꼴 모양}


% ========================================== chapter ============
\newpage  
\chapter{글자 색}

	

% ========================================== chapter ============
\newpage  
\chapter{윗첨자, 아래첨자}

																																						
																																								
																																										
																																												
							








% ========================================== chapter ============
\newpage  
\chapter{글꼴}

	% ------------------------------------------ section ------------ 
	\newpage  \null
	\section{글꼴}

			ko.TeX에서 기본 자원하는 기본 글꼴 목록 표\ref{gibonFont} 과 
			기본 설정은 표\ref{gibonSetting}와 같다. 
			확장글꼴과 트루타입글꼴의 추가와 같이 더 자세한 사항은 ko.TeX 가이드 문서를 참조하라.
		
	\subsection{기본 글꼴}
				\vspace{-1cm}
				\begin{table}[hbp]
				\caption{ko.TeX기본 글꼴 목록}
				\centering
				\begin{tabular}{c c}
				\toprule
				글꼴 & 글꼴 이름\\
				\toprule
				명조	& utbt  \\
				고딕	& utgt \\
				타자	& uttz \\
				그래픽	& utgr \\
				\bottomrule
				\end{tabular}%
				\label{gibonFont}%
				\end{table}%
		
	\subsection{기본 설정}
				\vspace{-1cm}
				\begin{table}[hbp]
				\caption{ko.TeX 문서 한글 기본 설정}
				\centering
				\begin{tabular}{c c c c c}
				\toprule
				언어 종류 & rm 글꼴 이름 & sf 글꼴 이음 & tt 글꼴 이음 & emph 글꼴 이음 \\
				\toprule
				한글 & utbt & utgt & uttz & utgr \\
				한자 & utgt & utgt & uttz & utgr \\
				\bottomrule
				\end{tabular}%
				\label{gibonSetting}%
				\end{table}%

	% ------------------------------------------ section ------------ 
	\newpage  \null
	\section{글꼴}




			몇 가지 생각나는 대로 말씀드리면 다음과 같습니다. 
			우선, LaTeX은 "특정 폰트"에 대해서 무관심하다고 말할 수 있습니다. 
			LaTeX이 유지하고자 하는 것은 문서의 논리적 일관성이지 
			외형상의 모양이 아니므로(그것이 중요하지 않다는 뜻은 아닙니다), 
			
			LaTeX으로 문서를 작성했을 때, 그것이 결과적으로 어떤 폰트로 식자될 것인가는 최
			종적인 편집자가 결정할 문제이지 문서 작성자가 신경쓸 일이 아니라고 생각한다고 
			볼 수 있습니다.
			
			그러므로, LaTeX에서는 
			이를테면, "세리프 글꼴", "산세리프 글꼴", "타이프라이터 글꼴"이라는 
			세 가지 기본적인 폰트를 하나의 문서에서 사용하는 것으로 일단 상정합니다. 
			각각 한글폰트로는 명조체, 고딕체, 타자체 정도에 해당하는 것이 아닌가 싶습니다만, 
			1:1 대응이 되는지는 잘 모르겠습니다. 
			이 세 가지를 각각 rm, sf, tt라고  부르고 "글꼴 가족"(font family)이라고 합니다.
			 주1. 이 이외의 글꼴 가족을 더 정의해서 쓰는 것은 물론 가능합니다.
			
			각각의 글꼴 가족에 대해서 
			모양(shape)과 계열(series)이라는 추가적인 옵션이 있습니다. 
			모양에는 itshape(이탤릭), slshape(기울인 모양), upshape(곧게선 모양), 
			scshape(작은대문자모양) 등이 있고, 
			계열에는 mdseries(보통글꼴 계열), bfseries(굵은 계열), extended(늘린 계열), 
			condensed(줄인 계열) 등이 있습니다.\\
			
			\textbf{모양(shape)}\\
			\begin{itemize}[topsep=-1.0em, itemsep=-0.5em]
			\item	itshape	(이탤릭)
			\item	slshape	(기울인 모양)
			\item	upshape	(곧게선 모양)
			\item	scshape	(작은대문자모양)
			\end{itemize}
			
			
			\paragraph*{계열(series)}
			\begin{itemize}[topsep=-1.0em, itemsep=-0.5em]
			\item	mdseries	(보통글꼴 계열)
			\item	bfseries		(굵은 계열)
			\item	extended	(늘린 계열)
			\item	condensed	(줄인 계열)
			\end{itemize}
			
			 
			LaTeX은 이 명령들을 결합해서 폰트를 지정합니다. 
			예를 들면, \verb|\bfseries|라고 하면, 
			현재의 폰트 모양에서 계열만을 bfseries로 바꾸라는 뜻입니다. 
			이 선언은 한번 바뀌면 이후로도 계속 영향을 미치기 때문에, 
			\verb|\textbf{ABCDEFghi}|와 같은 식으로 일부에 대해서만 적용하도록 하는 
			\verb|\text...| 명령이 존재합니다.
			
			아무런 지정도 하지 않은 상태에서는 \verb|\normalfont|가 적용되는데, 
			일반적으로 serif, upshape, mdseries가 기본 글꼴입니다. \\
			
			이제 이것을 특정 폰트와 결합해야 합니다. TeX/LaTeX은 시스템의 폰트를 이용하지 
			않습니다. 사실, 엄격히 말하자면 LaTeX이 할 수 있는 일은, 최종 출력기에게 "어떠
			어떠한 폰트를 사용해서 이 글자를 어느 위치에 식자하라"는 명령만을 할 수 있을 뿐
			이고, 폰트 자체를 사용하지 않습니다.\\
			
			 주2. TeX의 출력포맷인 .dvi에는 폰트에 대한 "정보"만이 들어 있고 폰트는 들어 있
			지 않습니다.\\
			
			TeX이 식자에 사용하는 정보는 .tfm이라는 Font Metric 파일에 있습니다. 
			다시 말하면 .tfm만을 TeX이 이해한다고 할 수 있지요. 
			말씀하신 Arial이나 Times New Roman 같은 것은 .tfm 형태로 존재하지 않는 한 
			TeX이 사용할 수 없습니다.
			
			TeX이 기본으로 제공하는 영문폰트는 TeX의 창시자인 D. Knuth가 디자인한 Computer 
			Modern이라는 글꼴 계열입니다.
			이 글꼴은 예컨대, 다음과 같은 방법으로 이 폰트를 결합시킵니다.\\
			
			\verb|\rmfamily| => Computer Modern Roman\\
			\verb|\sffamily| => Computer Modern Sans Serif\\
			\verb|\ttfamily| => Computer Modern Typewriter\\
						
			같은 Computer Modern Roman 글꼴이라 해도, 각각의 shape와 series에 따라서 (심지
			어 글꼴 크기에 따라서도) 별도의 .tfm이 준비되어 있습니다.
			
			---
			LaTeX의 발달에 따라서 다른 폰트도 문서에 사용할 수 있게 하려는 노력이 있었습니
			다. 
			예를 들면 범용의 35개의 LaserWriter Postscript Font를 쓸 수 없는가? 이를 위
			해서 각각의 글꼴 가족에 cm이 아닌 다른 폰트를 대응시켜 사용할 수 있습니다.\\
			
			우리는 일반적으로 Times라고 쉽게 말하지만, 사실 Adobe Times는 아무나 사용할 수 
			있는 공개 글꼴이 아니고 저작권이 존재하는 상업용 글꼴입니다. 이 Times를 흉내낸 
			(가짜) times를 자유롭게 사용할 수 있을 뿐인데, 대표적인 것은 GhostScript와 함
			께 제공되는 urw 폰트입니다. (윈도 시스템은 그 나름대로 글꼴을 라이센싱해서 제공
			하므로, 사용할 수 있는 글꼴의 형태와 폭이 또 다르다고 할 수 있습니다.)
			이 urw times를 Computer Modern Roman 대신 사용하려면 약간 복잡한 절차를 거쳐서 
			urw times의 .tfm을 정의한 다음, \verb|\rmfamily|를 부르는 곳에서 해당 .tfm을 연결시켜
			주면 될 것입니다. 다행히 주요 Postscript 영문 글꼴에 대해서는 이런 작업이 TeX 
			Implementation에서 다 되어 있고, 사용자는 단순히 다음과 같이 한 줄을 .tex 파일
			의 preamble에 적어주면 됩니다.
			

			\verb|\usepackage{times}|

			그런데, 이것은 오로지 본문 글꼴만 바꾸는데, 수학 글꼴(TeX은 수학식에는 본문 폰
			트와는 별도의 폰트를 적용합니다.)에도 times 비슷한 폰트를 쓰도록 하려는 목적에
			서 요즘은
			
			\verb|\usepackage{mathptmx}|
			
			이런 식으로 할 것을 권장합니다.
			
			아무튼, (오래된 패키지라는 사람도 있지만) \verb|\usepackage{times}|하면, 
			\verb|\sffamily|에는 helvetica 폰트를 선택해줍니다.
			이밖에 times를 본문으로 하고 helvetica나 다른 sansserif(물론 공개 글꼴 중에서)
			를 선택하게 해주는 스타일 패키지 중에, 사용이 간편하고 품위도 나쁘지 않은 것으
			로는 txfonts라는 것이 있습니다\\. 
			
			\verb|\usepackage{txfonts}|\\
			
			이 스타일을 얹으면 rm에는 times와 비슷한 글꼴을, sf에는 helvetica 비슷하지만 크
			기가 helvet보다 조금 더 적절하고 모양도 다듬어진 글꼴을 사용할 수 있게 됩니다. 
			
			---
			마지막으로, Arial과 Times New Roman을 "반드시" 써야 하고, TeX이 기본적으로 제공
			하는 폰트에 불만이시라면, 이것은 (La)TeX의 폰트 선택 스킴을 공부하셔서 .tfm을 
			추출하고 스타일을 정의해서 쓰셔야 할 것입니다. 
			윈도 기본 폰트에 대해서는 이미 이런 작업을 해두신 분들이 아마도 있을 것입니다만, 
			TeX 배포판에 기본적으로 포함되어 있지는 않을 것으로 생각합니다.
			
			---
			더 쉽게 재미있게 설명해주실 분을 기대합니다.













































	% ------------------------------------------ section ------------ 
	\newpage  \null
	\section{글꼴 바꾸기}

			하나의 문서에 글꼴이 많이 쓰이면 쓰일수록 통일성이 떨어지고, 
			한눈에 읽기도 좋지 않다. 그러므로 되도록 글꼴을 유지할 것을 권한다.
			그럼에도 표준 글꼴이 마음에 들지 않는다면 다음의 명령어를 사용할 수 있다.
			\begin{framed}
			\verb|\SetHangulFonts{rm(roma)}{ss(san serif)}{tt(typewriter)}|
			\verb|\SetHanjaFonts{rm(roma)}{ss(san serif)}{tt(typewriter)}|
			\end{framed}
			위의 명령은 각각 한글과 한자 글꼴을 지정한다. 
		
			
	

	% --------------- subsection -------------- 글꼴 모양
	\clearpage  \null
	\section{글꼴 모양}
	
		기본적인 형식은 아래와 같으며, 한글의 경우 이탤릭체보다는 굵은 글씨를 쓰는 경우가 많다.
		
	
			\begin{table}[hbp]
			\caption{글자 모양}
			\centering
			\begin{tabular}{c c c}
			\toprule
			명령어 & 환경 & 결과 \\
			\toprule
			\verb|\textnormal|	& textnormal 	& 결과 \\
			\verb|\textit|		& itshape		& \Large\textit{결과} \\
			\verb|\emph| 		& 없음			& \Large\emph{결과} \\
			\verb|\textbf|		& bfseries	 	& \Large\textbf{결과} \\
			\verb|\underline| 	& underline 		& \Large\underline{결과} \\
			\bottomrule
			\end{tabular}%
			\end{table}%
	
	
			\begin{table}[hbp]
			\caption{글자 모양}
			\centering
			\begin{tabular}{c c c}
			\toprule
			명령어 & 환경 & 결과 \\
			\toprule
			\verb|\rm|	& Roman 	& \rm{결과 result} \\
			\verb|\sl|	& Slanted	& \sl{결과 result} \\
			\verb|\it| 	& Italic		& \it{결과 result} \\
			\verb|\tt|	& Typewrite	& \tt{결과 result} \\
			\verb|\bf| 	& Bold 		& \bf{결과 result} \\
			\bottomrule
			\end{tabular}%
			\end{table}%

	
		
	% --------------- subsection -------------- 밑줄 긋기
	\clearpage \null
	\section{밑줄 긋기}
	
		보통 밑줄을 사용하지 않는다. 혹 사용한다 하더라도 \verb|\underline|으로 충분하다. 
		그래도 꼭 다양한 밑줄을 그어야 한다면 ulem 패키지를 사용하면 된다.
















