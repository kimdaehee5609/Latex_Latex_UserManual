
%
%		\usepackage{pifont}		% 		
%		\usepackage{textcomp}		%	 		
%		\usepackage{gensymb}		% 		
%		\usepackage{marvosym}		% 		


% ========================================== chapter ============
	\chapter{특수문자}

	% -------------------------------------- page -------------------
	%	\nomtcrule         		% removes rules = horizontal lines
	%	\nomtcpagenumbers  % remove page numbers from minitocs
		\newpage
		\minitoc				% Creating an actual minitoc
	%	\doublespace



% ------------------------------------------ section ------------ 특수문자
	\section{특수 문자}

		\subsection{따옴표}
			타지기에서처럼 $''$를 사용해서는 안된다. 
			
			출판물에서는 특별한 여는 따옴표와 닫는 따옴표를 사용한다. 
			\LaTeX에서는 두개의 $'$ (grave accent)와 두개의 $'$(vertical quote)를 
			각각 여는 따옴표와 닫는 따옴표로 사용한다.
			작은 따옴표의 경우에는 각각의 기호를 한 번씩만 사용하면 된다.
		
			\begin{quote}
			(ㄱ)왼쪽 따옴표 (ㄱ)오른쪽 따옴표
			\end{quote}
		
			\paragraph{여는 따옴표}
			여는 따옴표는 표준 자판 제일 윗쪽의 제일 왼쪽 키에 할당되어 있는 문자
			(back-tick 혹은 grave accent)를 사용한다.
		
			\paragraph{닫는 따옴표}
			닫는 따옴표는 자판 가운데줄 오른쪽 끝에 있는 작은 따옴표 문자(vertical quote)를 사용한다.
		
		
		\subsection{대시와 히이픈}
				
						
		\subsection{틸데(\~{})}
	
		\subsection{도 기호($^{\circ}$)}
	
			\begin{framed}
			\verb|$30\,^{\circ} \mathrm{C} $|  	$30\,^{\circ} \mathrm{C} $ \\
			\verb|$30^{\circ} \mathrm{C} $|  		$30^{\circ} \mathrm{C} $
			\end{framed}
	
		\subsection{유로 통화 기호}
		
	
		\subsection{줄임표}




% ------------------------------------------ section ------------ 
	\section{특수문자}


	% --------------------------------------------------
	\subsection{ textcircled }
	\begin{itemize}[itemsep=-0.5em,leftmargin=4em]
	\item[	\textcircled {\footnotesize A} ] \textbackslash textcircled \{\textbackslash footnotesize A\}
	\item[	\textcircled {\footnotesize B} ] \textbackslash textcircled \{\textbackslash footnotesize B\}
	\item[	\textcircled {\footnotesize C} ] \textbackslash textcircled \{\textbackslash footnotesize C\}
	\item[	\textcircled {\footnotesize D} ] \textbackslash textcircled \{\textbackslash footnotesize D\}
	\item[	\textcircled {\footnotesize E} ] \textbackslash textcircled \{\textbackslash footnotesize E\}
	\item[	\textcircled {\footnotesize F} ] \textbackslash textcircled \{\textbackslash footnotesize F\}
	\item[	\textcircled {\footnotesize G} ] \textbackslash textcircled \{\textbackslash footnotesize G\}


	\item[	\textcircled {\footnotesize 1} ] \textbackslash textcircled \{\textbackslash footnotesize 1\}
	\item[	\textcircled {\footnotesize 2} ] \textbackslash textcircled \{\textbackslash footnotesize 2\}
	\item[	\textcircled {\footnotesize 3} ] \textbackslash textcircled \{\textbackslash footnotesize 3\}

	\item[	\textcircled {\footnotesize 11}] \textbackslash textcircled \{\textbackslash footnotesize 11\}
	\item[	\textcircled {\scriptsize   11}] \textbackslash textcircled \{\textbackslash scriptsize   11\}
	\item[	\textcircled {\tiny         11}] \textbackslash textcircled \{\textbackslash tiny         11\}


	\item[	\textcircled {\footnotesize 가}] \textbackslash textcircled \{\textbackslash footnotesize 가\}
	\item[	\textcircled {\scriptsize   가}] \textbackslash textcircled \{\textbackslash scriptsize   가\}
	\item[	\textcircled {\tiny         가}] \textbackslash textcircled \{\textbackslash tiny         가\}

	\end{itemize}

	% --------------------------------------------------
	\subsection{gensymb}

	\celsius \\
	\perthousand \\
	\degree	\\

	\textbullet ~ textbullet

	\textzerooldstyle\\

	$\div$\\

	% --------------------------------------------------
	\subsection{textcomp}

			\verb|\usepackage{textcomp}|
			

	\begin{itemize}
		\item	\textcelsius  		\verb|\textcelsius|
		\item	\textpertenthousand	\verb|\textpertenthousand|
		\item	\textperthousand		\verb|\textperthousand|
		\item	\textreferencemark		\verb|\textreferencemark|
	\end{itemize}





	\newpage \null
	\subsection{Table 211: marvosym Engineering Symbols}

			\verb|\usepackage{marvosym}|


	\begin{itemize}
		\item	\Beam		\verb|	\Beam	|
		\item	\Force		\verb|	\Force	|
		\item	\Octosteel		\verb|	\Octosteel	|
		\item	\RoundedTTsteel		\verb|	\RoundedTTsteel	|
		\item	\Bearing		\verb|	\Bearing	|
		\item	\Hexasteel		\verb|	\Hexasteel	|
		\item	\Rectpipe		\verb|	\Rectpipe	|
		\item	\Squarepipe		\verb|	\Squarepipe	|
		\item	\Circpipe		\verb|	\Circpipe	|
		\item	\Lefttorque		\verb|	\Lefttorque	|
		\item	\Rectsteel		\verb|	\Rectsteel	|
		\item	\Squaresteel•		\verb|	\Squaresteel•	|
		\item	\Circsteel		\verb|	\Circsteel	|
		\item	\Lineload		\verb|	\Lineload	|
		\item	\Righttorque		\verb|	\Righttorque	|
		\item	\Tsteel		\verb|	\Tsteel	|
		\item	\Fixedbearing		\verb|	\Fixedbearing	|
		\item	\Loosebearing		\verb|	\Loosebearing	|
		\item	\RoundedLsteel		\verb|	\RoundedLsteel	|
		\item	\TTsteel		\verb|	\TTsteel	|
		\item	\Flatsteel		\verb|	\Flatsteel	|
		\item	\Lsteel		\verb|	\Lsteel	|
		\item	\RoundedTsteel		\verb|	\RoundedTsteel	|
	\end{itemize}




%	\subsection{\asterisus asterisus}










