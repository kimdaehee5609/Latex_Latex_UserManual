\documentclass[12pt,a4paper]{report}
\usepackage{kotex}
\usepackage[unicode]{hyperref}	% 한굴 하이퍼링크 사용

\usepackage{enumerate}			%
\usepackage{enumitem}			%

\usepackage{setspace}
\usepackage{amssymb,amsfonts,amsmath}

\usepackage{booktabs}
\usepackage[table]{xcolor}
\usepackage{rotating}
\usepackage{array}
\usepackage{tabularx}
\usepackage{tabulary}
\usepackage{multirow}

\usepackage{framed} % 글상자


% ----------------------------- 장의 목차
\usepackage{minitoc}
	\setcounter{minitocdepth}{1}    	% Show until subsubsections in minitoc
	\setlength{\mtcindent}{12pt} 	% default 24pt


\begin{document}

		\title{수식}
		\maketitle

		% ----------------------- 장 목차 생성 Initialization
		\dominitoc		
		
		\newpage
		\tableofcontents
		
		
		\listoffigures
		\listoftables
		
		\doublespace
		\setcounter{secnumdepth}{3}
		\setlength{\parindent}{0cm}
		
		
% ========================================== chapter ============
\newpage
\chapter{수식 관련 package}
	
		% ----------------------- 장 목차 생성 Initialization
		\minitoc		
		
% -------------------------------------------------------------
%
%
%
% -------------------------------------------------------------
\newpage
\section{수식 관련 package}

\newpage
\section{AMS package}




\newpage
\section{ams math package}



\newpage
\section{ams open package}


\newpage
\section{ams text package}

	\begin{itemize}[itemsep=0.0em]
	\item	$\backslash$text 명령은 하위 패키지인  amstext package를 통해 amsmath package에서 정의된다
	\item	amstext package를 독립적으로 사용할 수도 있다.
	\item	$\backslash$text 명령의 주요 사용은 화면에서 단어나 문구입니다.
	\item	LATEX 명령의 $\backslash$MBOX와  효과가 매우 유사하지만 몇 가지 장점이있다.
			당신이 첨자 입력을 단어나 문장으로 하려는 경우에는 $\backslash$MBOX를 사용하는것 보다 
			약간 쉽게 \begin{verbatim}{\text{word or phrase}} \end{verbatim} 입력 할 수 있습니다 

	\item 또 다른 장점은 더 설명이 포함 된 이름이다.	
	\end{itemize}
	
	\begin{framed}
	\begin{verbatim}
		f_{[x_{i-1},x_i]} \text{ is monotonic,}\quad i = 1,\dots,c+1
	\end{verbatim}
	\end{framed}
	
	\begin{equation}
		f_{[x_{i-1},x_i]} \text{ is monotonic,}\quad i = 1,\dots,c+1
	\end{equation}	
	
		
		
		
\newpage
\section{ams bsy package}


\newpage
\section{ams intx package}


\newpage
\section{ams cd package}



\newpage
\section{ams thm package}


\newpage
\section{AMS document classes}






% -------------------------------------------------------------
%
%
%
% -------------------------------------------------------------
\newpage
\section{수식의 입력 개요}

	LATEX는 수학식 조판을 위한 특별한 모드를 갖고 있다.
	수학식은 단락 안에서 행중 수식으로 조판될 수도 있고, 별도의 단락으로 조판될 수도 있다.

	\paragraph{행 중 텍스트 방식}
	단란안에서 행중 수식 텍스트는 \verb|\(|와 \verb|\)|, \verb|$|와 \verb|$|, 또는 
	\verb|\begin{math}|와 \verb|\end{math}| 사이에 들어간다.

	\begin{itemize}
	\item 	\verb|\(| 수학식 \verb|\)|
	\item 	\verb|$| 수학식 \verb|$|
	\item 	\verb|\begin{math}| 수학식 \verb|\end{math}|
	\end{itemize}


	\paragraph{별도 단락으로 표시}
	별도 단락으로 큰 수학 기호를 사용하는 방정식이나 공식을 식자하려면
	실제로 단락을 나누는 것보다는 수식 보여주기 하는 것이 바람직하다.
	이렇게 하기 위해서는, 이들을 \verb|\[|와 \verb|\]|안에 넣거나, 
	\verb|\begin{displaymath}|와 \verb|\edn{displaymath}| 사이에 넣는다.

	\begin{itemize}
	\item 	\verb|\[| 수학식 \verb|\]|
	\item 	\verb|\begin{displaymath}| 수학식 \verb|\edn{displaymath}|
	\end{itemize}


	\paragraph{별도 단락으로 표시하고 수식 번호를 붇이는 경우}
	수식 번호를 \LaTeX 가 붙여 주기를 원한다면, equation 환경을 쓸 수 있다. 
	그러면 수식 번호에 \verb|\label|을 붙이고 \verb|\ref| 또는 
	amsmath  패키지에 있는 \verb|\eqref| 명령을 써서 본문 안 다른 곳에서 이 레이블을 참조할 수 있다.

	\begin{itemize}
	\item 	\verb|\begin{equation}| 수학식 \verb|\edn{equation}|
	\end{itemize}


% -------------------------------------------------------------
%
%
%
% -------------------------------------------------------------
\clearpage
\section{그리스 소문자의 입력}
	\begin{table}[hbp]
		\caption{그리스 소문자 입력}
		\centering
		\begin{tabular}{ l l }
		\toprule
		$\alpha$		&\verb|\alpha| \\
		$\beta$			&\verb|\beta|  \\ 
		γ 				&\verb|\gamma|  \\
		δ 				&\verb|\delta|  \\ 
		$\epsilon$		&\verb|$\epsilon$|  \\
		ε				&\verb|\varepsilon| \\
		ζ 				&\verb|\zeta| \\ 
		η 				&\verb|\eta|  \\
		θ 				&\verb|\theta| \\
		$\vartheta$ 	&\verb|\vartheta| \\
		ι				&\verb| \iota|  \\		 
		κ 				&\verb|\kappa| \\		
		λ				&\verb|\lambda| \\		
		μ				&\verb|\mu|	 \\		
		ν				&\verb|\nu|	 \\		
		ξ 				&\verb|\xi| \\			
		o				&\verb|o| \\		
		$\pi$			&\verb|\pi| \\		
		$\varpi$	 	&\verb|\varpi| \\	
		ρ				&\verb| \rho| \\	
		$\varrho$		&\verb|\varrho| \\	
		σ		&\verb|\sigma| \\	
		$\varsigma$	&\verb|\varsigma| \\
		τ		&\verb|\tau|  \\
		υ 	&\verb|\upsilon| \\
		φ		&\verb|\phi| \\
		$\varphi$	 	&\verb|\varphi| \\
		χ		&\verb|\chi| \\
		ψ		&\verb|\psi| \\
		ω		&\verb|\omega| \\
	    \bottomrule
		\end{tabular}%
	\end{table}%


	
% -------------------------------------------------------------
%
%
%
% -------------------------------------------------------------
\clearpage
\section{수식의 배치}


	\subsection{수식에서 여러줄 배치 및 맞춤}
		\paragraph{eqnarray 환경}
		여러 줄에 걸친 식이나 수식군을 위해서는 equation  대신 eqnarray 환경을, equation* 대신 eqnarray*를 쓸 수 있다.
		
		\paragraph{수식 번호}
		eqnarray 안에서 각 줄마다 하나씩 수식번호가 붙는 다. eqnarray*는 번호가 붙지 않는다.
		
		\paragraph{자리 맞춤}
		equnarray와 equnarray* 환경은 \verb|{rcl}| 형식의 세 컬럼 표(tabular)와 같이 동작하며,
		가운데 칸은 등호, 부등호, 그 밖에 적당한 다른 기호를 두는데 쓴다.

		\paragraph{줄 바꿈}
		\verb|\\| 명령은 줄 나눔을 표시한다

		\clearpage
		\paragraph{사용예}
		\begin{verbatim}
			\begin{eqnarray}
			    f(x) & = & \cos x \\
    			    f`(x) & = & - \sin x \\
			    \int_{0}^{x} f(x) dy & = & \sin x 
			\end{eqnarray}
		\end{verbatim}
		\paragraph{결과}
		\begin{eqnarray}
			f(x) & = & \cos x \\
			f`(x) & = & - \sin x \\
			\int_{0}^{x} f(x) dy & = & \sin x 
		\end{eqnarray}
		
		\paragraph{등호 양쪽 간격 조정}
		등호 양쪽 간격아 좀 넓어 보인다는 것을 알보 두자. 이를 고치려면 \verb| \set length \ array col sep {2pt}|로 설정하면 이를 고칠 수 있다.
		
	
		\paragraph{결과}
		\setlength\arraycolsep{2pt}
		\begin{eqnarray}
			f(x) & = & \cos x \\
			f`(x) & = & - \sin x \\
			\int_{0}^{x} f(x) dy & = & \sin x 
		\end{eqnarray}
		
% -------------------------------------------------------------
%
%
%
% -------------------------------------------------------------
\clearpage
\section{긴 수식의 표현}


	긴 수식은 자동으로 적절하게 나누어지지 않을 것이다. 
	저자는 어디서 나눌것인지 들여쓰기를 얼마나 할지 지정해야 한다. 
	이를 위해서 다음 두 방법이 가장 널리 쓰인다.
	\\
		% -----------------------------------------
		\rule{\linewidth}{1pt}
		\paragraph{사용예}
		\begin{verbatim}
			\setlength\arraycolsep{2pt}ds
			\begin{eqnarray}
			    \sin x 	& = & x -\frac{x^{3}}{3!} + \frac{x^{5}}{5!} - {}
			    \nonumber \\
					    	& & {}-\frac{x^{7}{7!}+{} \cdots
			\end{eqnarray}
		\end{verbatim}
		% -----------------------------------------
		\rule{\linewidth}{1pt}
		\paragraph{결과}
			\setlength\arraycolsep{2pt}
			\begin{eqnarray}
			    \sin x 	& = & x -\frac{x^{3}}{3!} + \frac{x^{5}}{5!} - {}
			    \nonumber \\
					 	& & {}-\frac{x^{7}}{7!}+{} \cdots
			\end{eqnarray}
		\rule{\linewidth}{1pt}
	
	\verb|\nonumber| 명령은 LATEX에게 이 수식에 번호를 붙이지 말라고 알려준다.
	


% -------------------------------------------------------------
%
%
%
% -------------------------------------------------------------
\clearpage
\section{기본 수식의 입력}

	\subsection{윗첨자 아래첨자}

	\subsection{사칙연산}

	\subsection{제곱근}
	
	\subsection{기본함수}
	
	one \hfill two\\
	
	
		\begin{tabbing}
		~\hspace{3cm}\= ~\hspace{3cm} \= ~\hspace{3cm} \= ~\hspace{3cm} \= \kill \\
		\verb|\sin| \> \verb|\cos| \> \verb|\tan| \> \\
		\verb|\log| \> \verb|\max| \> \verb|\min| \>
		\end{tabbing}
	
	


% -------------------------------------------------------------
%
%
%
% -------------------------------------------------------------
\clearpage
\section{Arrow symbols}

			\textbf{Left}\\
			\begin{tabular}{ p{0.1\textwidth} p{0.3\textwidth} 
							p{0.1\textwidth} p{0.3\textwidth}  }
			\toprule
			$\leftarrow$		&$\backslash$leftarrow &
			$\longleftarrow$	&$\backslash$longleftarrow \\
			$\Leftarrow$		&$\backslash$Leftarrow&
			$\Longleftarrow$	&$\backslash$Longleftarrow \\
			\bottomrule
			\end{tabular} \\
			
			\textbf{Right}\\
			\begin{tabular}{ p{0.1\textwidth} p{0.3\textwidth} 
							p{0.1\textwidth} p{0.3\textwidth}  }
			\toprule
			$\rightarrow $		&$\backslash$rightarrow& 
			$\longrightarrow$	&$\backslash$longrightarrow \\
			$\Rightarrow$		&$\backslash$Rightarrow&
			$\Longrightarrow$	&$\backslash$Longrightarrow \\
			\bottomrule
			\end{tabular} \\
		
			\textbf{Left-Right}\\
			\begin{tabular}{ p{0.1\textwidth} p{0.3\textwidth} 
							p{0.1\textwidth} p{0.3\textwidth}  }
			\toprule
			$\leftrightarrow $		&$\backslash$leftrightarrow& 
			$\longleftrightarrow$	&$\backslash$longleftrightarrow \\
			$\Leftrightarrow$		&$\backslash$Leftrightarrow&
			$\Longleftrightarrow$	&$\backslash$Longleftrightarrow \\
			\bottomrule
			\end{tabular} \\

			\textbf{Up-Down}\\
			\begin{tabular}{ p{0.1\textwidth} p{0.3\textwidth} 
							p{0.1\textwidth} p{0.3\textwidth}  }
			\toprule
			$\uparrow $			&$\backslash$uparrow& 
			$\Uparrow$				&$\backslash$Uparrow \\
			$\downarrow$			&$\backslash$downarrow&
			$\Downarrow$			&$\backslash$Downarrow\\
			$\updownarrow$		&$\backslash$updownarrow&
			$\Updownarrow$		&$\backslash$Updownarrow\\
			\bottomrule
			\end{tabular} \\

			\textbf{arrow}\\
			\begin{tabular}{ p{0.1\textwidth} p{0.3\textwidth} 
							p{0.1\textwidth} p{0.3\textwidth}  }
			\toprule
			$\nearrow $				&$\backslash$nearrow \\
			$\searrow $				&$\backslash$searrow \\ 
			$\swarrow $			&$\backslash$swarrow \\ 
			$\nwarrow $			&$\backslash$nwearrow \\ 
			\bottomrule
			\end{tabular} \\





































\end{document}
