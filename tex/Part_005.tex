
			


	
			
% ========================================== chapter ============================
\newpage
\chapter{편집 용지}

	% -------------------------------------- page -------------------
		\minitoc				% Creating an actual minitoc


% ------------------------------------------ section ------------ 
\newpage
\section{판형 (페이지)}

	\subsection{용지 크기}
	
		\begin{itemize}
			\item	a4paper, 
			\item	a5paper, 
			\item	b5paper, 
			\item	letterpaper, 
			\item	legalpaper,
			\item	executivepaper
		\end{itemize}


% -------------------------------------------------------------
\newpage
\section{판면 (편집영역)}

% -------------------------------------------------------------
\newpage
\section{면주와 여백문단)}

	
	
% ------------------------------------------ section ------------ 
\section{문서 여백}

	
		\setlength{\fboxsep}{12pt}
		\begin{boxedminipage}[c]{1.0\linewidth}
			\begin{verbatim}
			\set length { 편집할 여백 } {크기}
			\set length { \left margin } {2cm}
			\end{verbatim} 
		\end{boxedminipage}
		


	\paragraph{ geometry package} \hfill

%		\setlength{\fboxsep}{12pt}
		\begin{boxedminipage}[c]{1.0\linewidth}
			\begin{verbatim}
			\ use package 
				[left=1.0cm, right=1.0cm, top=1.0cm, bottom=1.0cm] { geo metry }
			\end{verbatim}
		\end{boxedminipage}

	%	------------------------------------------------ code	
		\begin{mdframed}[style=code_document, frametitle={code}]
			\begin{verbatim}
			\usepackage{geometry}
			\geometry{landscape=true	}
			\geometry{top 		=10em}
			\geometry{bottom		=10em}
			\geometry{left		=8em}
			\geometry{right		=8em}
			\geometry{headheight	=4em} % 머리말 설치 높이
			\geometry{headsep		=2em} % 머리말의 본문과의 띠우기 크기
			\geometry{footskip		=4em} % 꼬리말의 본문과의 띠우기 크기
	 		\geometry{showframe}
		
	%		paperwidth 	= left + width + right (1)
	%		paperheight 	= top + height + bottom (2)
	%		width 		= textwidth (+ marginparsep + marginparwidth) (3)
	%		height 		= textheight (+ headheight + headsep + footskip) (4)
			\end{verbatim}
		\end{mdframed}





	% -------------------------------------- page -------------------
	\newpage  
	\subsection*{문서 여백 : set length }

		\singlespacing
		\setlength{\fboxsep}{12pt}
		\begin{boxedminipage}[c]{1.0\linewidth}
			\begin{verbatim}
			\set length { \편집할 여백 } { 크기 }
			
			\set length { \left margin } { 2cm }
			\set length { \right margin } { 2cm }
			\set length { \odd side margin } { 2cm }  % 홀수쪽
			\set length { \even side margin } { 2cm } % 짝수쪽
			\set length { \top margin } { 2cm }
			\set length { \text width } { 18cm }
			\set length { \text height } { 25cm }
			\end{verbatim} 
		\end{boxedminipage}
		\doublespacing

	% -------------------------------------- page -------------------
	\newpage  
	\subsection*{문서 여백}

		\singlespacing
		\setlength{\fboxsep}{12pt}
		\begin{boxedminipage}[c]{1.0\linewidth}
			\begin{verbatim}
				1. one inch + \hoffset
				2. one inch + \voffset
				
				\oddsidemargin = 31pt
				\topmargin = 20pt
				\headheight = 12pt   % 머리말 높이
				\headsep = 25pt       % 머리말과 본문사이 간격
				
				\textheight = 592pt  % 본문의 높이
				\textwidth = 390pt	% 본문의 폭
				
				\marginparsep = 10pt
				\marginparwidth = 35pt
				
				\footskip = 30pt  % 본문과 꼬리말 사이 간격
				
				\marginparpush = 7pt (not shown)
				\hoffset = 0pt
				\voffset = 0pt
				
				\paperwidth = 597pt
				\paperheight = 845pt
			\end{verbatim} 
		\end{boxedminipage}
		\doublespacing


	% -------------------------------------- page -------------------
	\newpage  
	\subsection*{문서 여백 : geometry }

		\singlespacing
		\setlength{\fboxsep}{12pt}
		\begin{boxedminipage}[c]{1.0\linewidth}
			\begin{verbatim}
			\use package { geo metry }
			
			\geometry { paper size = { 25cm, 35cm }
			\geometry { total = { 20cm, 30cm }
			\geometry { body = { 18cm, 25cm }
			\geometry { hmargin = { 3cm, 2cm }
			\geometry { vmargin = { 2cm, 3cm }
			\geometry { margin par width = 2cm }
			\geometry { head = 1cm }
			\end{verbatim} 
		\end{boxedminipage}
		\doublespacing
	
		\singlespacing
		\setlength{\fboxsep}{12pt}
		\begin{boxedminipage}[c]{1.0\linewidth}
			\begin{verbatim}
			\use package 
			[ left=1.0in, right=1.0in, top=1.0in, bottom=1.0in ]
			{ geo metry }
			\end{verbatim} 
		\end{boxedminipage}
		\doublespacing












% ========================================== chapter ============================
\chapter{머리말 꼬리말}



% -------------------------------------- page -------------------
\section{머리말}


% -------------------------------------- page -------------------
\section{꼬리말}

	




% ========================================== chapter ============================
\chapter{쪽 번호}
	
	% -------------------------------------- page -------------------
	\newpage
	\section{쪽 번호}
	
	
% ========================================== chapter ============================
\chapter{쪽 나누기}
	

	
	% -------------------------------------- page -------------------
	\section{쪽 나누기}
	
		\begin{enumerate}[ topsep=0.0em, itemsep=-0.5em ]
		\item	\verb|\\| \\
				start a new paragraph.
		\item	\verb|\\*| \\
				 start a new line but not a new paragraph.
		\item	\verb|\- OK| \\
				to hyphenate a word here.
		\item	\verb|\cleardoublepage| \\
				flush all material and start a new page, start new odd numbered page.
		\item	\verb|\clearpage| plush all material and start a new page.
		\item	\verb|\hyphenation| enter a sequence pf exceptional hyphenations.
		\item	\verb|\linebreak| allow to break the line here.
		\item	\verb|\newline| request a new line.
		\item	\verb|\newpage| request a new page.
		\item	\verb|\nolinebreak| no line break should happen here.
		\item	\verb|\nopagebreak| no page break should happen here.
		\item	\verb|\pagebreak| encourage page break.
		\end{enumerate}
	
	



% ========================================== chapter ============================
\newpage
\chapter{줄 나누기}
		
	% -------------------------------------- page -------------------
	\newpage
	\section{줄 나누기}
	

% ========================================== chapter ============================
\newpage
\chapter{난 외주}

	% --------------------------------------- section -------------- 난 외주
	\newpage
	\section{난 외주}
	
		굉장히 쉽다\\
		\verb|\marginpar{text}|를 쓴다.\\
	
		\marginpar{난 외주 사용예}





























































