\documentclass[12pt,a4paper]{book}

% --------------------------------- 페이지 스타일 지정
	\usepackage{geometry}
	\geometry{hmargin={1.4in,1.4in}}
	\setlength{\topmargin}{00pt} 	%
	\setlength{\headsep}{45pt} 	%


		\usepackage{kotex}				% 한글 사용
		\usepackage[unicode]{hyperref}			% 한굴 하이퍼링크 사용
		\usepackage{amssymb,amsfonts,amsmath}% 수학 수식 사용
		
		\usepackage{enumerate}			%
		\usepackage{enumitem}			%
		\usepackage{pifont}				%
		\usepackage{setspace}			%
		\usepackage{booktabs}			% table
		\usepackage{color}				%
		\usepackage{multirow}			%
		\usepackage{boxedminipage}		% 미니 페이지
		\usepackage[pdftex]{graphicx}	% 그림 사용
		\usepackage[final]{pdfpages}	%pdf 사용
		\usepackage{framed}			%pdf 사용
		
		\usepackage{fix-cm}	
		\usepackage[english]{babel}

		\usepackage{tikz}%
		\usetikzlibrary{arrows,positioning,shapes}
		%\usetikzlibrary{positioning}


% ----------------------------- 장의 목차
\usepackage{minitoc}
	\setcounter{minitocdepth}{1}    	% Show until subsubsections in minitoc
	\setlength{\mtcindent}{12pt} 	% default 24pt

% --------------------------------- 페이지 스타일 지정

	\usepackage[Bjornstrup]{fncychap}

	\usepackage{fancyhdr}
	\pagestyle{fancy}
	\fancyhead{} % clear all fields
	\fancyhead[LO]{\small \leftmark}
	\fancyhead[RE]{\small \leftmark}
	\fancyfoot{} % clear all fields
	\fancyfoot[LE,RO]{\large \thepage}
	%\fancyfoot[CO,CE]{\empty}
	\renewcommand{\headrulewidth}{0.4pt}
	\renewcommand{\footrulewidth}{0.4pt}

% --------------------------------- recoment

	\newcommand{\red}{\color{red}}			% 글자 색깔 지정
	\newcommand{\blue}{\color{blue}}		% 글자 색깔 지정
	\newcommand{\black}{\color{black}}		% 글자 색깔 지정
	\newcommand{\superscript}[1]{${}^{#1}$}
	
	
% --------------------------------- 문서 기본 사항 설정

		\setcounter{secnumdepth}{1} % 문단 번호 깊이
		\setcounter{tocdepth}{1} % 문단 번호 깊이
		\setlength{\parindent}{0cm} % 문서 들여 쓰기를 하지 않는다.
		\doublespace
	
% --------------------------------- 환경 정의 : 박스 치고 안의 글자 빨간색

			\newenvironment{BoxRedText}
			{ 	\setlength{\fboxsep}{12pt}
				\begin{boxedminipage}[c]{1.0\linewidth}
				\color{red}
			}
			{ 	\end{boxedminipage} 
				\color{black}
			}
			


% ------------------------------------------------------------------------------
% Begin document (Content goes below)
% ------------------------------------------------------------------------------

	\begin{document}
		\dominitoc
		

% ------------------------------------------------------------------------------
% Maketitle
% ------------------------------------------------------------------------------
			\begin{titlepage}
			\thispagestyle{empty}				% Remove page numbering on this page
			\definecolor{grey}{rgb}{0.9,0.9,0.9} 
			\colorbox	{grey}
						{ \parbox[t]{1.0\linewidth}
						{
						\vspace*{1.2cm} 
						\fontsize{20}{20} \rmfamily \hfill \LaTeX 		\\ [0.8cm] \null
						\fontsize{40}{20} \rmfamily \hfill Page Layout \\ [0.8cm] \null
						\fontsize{20}{50} \rmfamily \hfill ver101
						\vspace*{0.8cm} 
						} }
			\vfill
			% Print the author data as defined above
			\hfill Kim Dae Hee\\ \null
			\hfill (주)서영엔지니어링\\ \null
			\hfill 건설관리팀\\ \null
			\hfill \url{h01038395609@gmail.com} \\ \null
			\hfill \rule{0.4\linewidth}{1pt}
			\end{titlepage}
			\cleardoublepage
% ------------------------------------------------------------------------------
% 
% ------------------------------------------------------------------------------


			\newpage
			\tableofcontents
			
			\newpage
			\listoffigures
			\listoftables
			

% \\\\\\\\\\\\\\\\\\\\\\\\\\\\\\\\\\\\\\\\\\\\\\\\\\\\\\\\\\\\\\\\\\\\\\\\\\\\\  Page layout

	\part{ Page Layout}
			
% ========================================== chapter ============================
\newpage
\chapter{편집 용지}

	% -------------------------------------- page -------------------
		\minitoc				% Creating an actual minitoc


% ------------------------------------------ section ------------ 
\newpage
\section{판형 (페이지)}

	\subsection{용지 크기}
	
		\begin{itemize}
			\item	a4paper, 
			\item	a5paper, 
			\item	b5paper, 
			\item	letterpaper, 
			\item	legalpaper,
			\item	executivepaper
		\end{itemize}


% -------------------------------------------------------------
\newpage
\section{판면 (편집영역)}

% -------------------------------------------------------------
\newpage
\section{면주와 여백문단)}

	
	
% ------------------------------------------ section ------------ 
\newpage  \null
\section{문서 여백}

	
		\setlength{\fboxsep}{12pt}
		\begin{boxedminipage}[c]{1.0\linewidth}
			\begin{verbatim}
			\set length { 편집할 여백 } {크기}
			\set length { \left margin } {2cm}
			\end{verbatim} 
		\end{boxedminipage}
		
		
		\textbf{ geo metry package }\\
		\setlength{\fboxsep}{12pt}
		\begin{boxedminipage}[c]{1.0\linewidth}
			\begin{verbatim}
			\ use package 
				[left=1.0cm, right=1.0cm, top=1.0cm, bottom=1.0cm] { geo metry }
			\end{verbatim}
		\end{boxedminipage}



	% -------------------------------------- page -------------------
	\newpage  
	\subsection*{문서 여백 : set length }

		\singlespacing
		\setlength{\fboxsep}{12pt}
		\begin{boxedminipage}[c]{1.0\linewidth}
			\begin{verbatim}
			\set length { \편집할 여백 } { 크기 }
			
			\set length { \left margin } { 2cm }
			\set length { \right margin } { 2cm }
			\set length { \odd side margin } { 2cm }  % 홀수쪽
			\set length { \even side margin } { 2cm } % 짝수쪽
			\set length { \top margin } { 2cm }
			\set length { \text width } { 18cm }
			\set length { \text height } { 25cm }
			\end{verbatim} 
		\end{boxedminipage}
		\doublespacing

	% -------------------------------------- page -------------------
	\newpage  
	\subsection*{문서 여백}

		\singlespacing
		\setlength{\fboxsep}{12pt}
		\begin{boxedminipage}[c]{1.0\linewidth}
			\begin{verbatim}
				1. one inch + \hoffset
				2. one inch + \voffset
				
				\oddsidemargin = 31pt
				\topmargin = 20pt
				\headheight = 12pt   % 머리말 높이
				\headsep = 25pt       % 머리말과 본문사이 간격
				
				\textheight = 592pt  % 본문의 높이
				\textwidth = 390pt	% 본문의 폭
				
				\marginparsep = 10pt
				\marginparwidth = 35pt
				
				\footskip = 30pt  % 본문과 꼬리말 사이 간격
				
				\marginparpush = 7pt (not shown)
				\hoffset = 0pt
				\voffset = 0pt
				
				\paperwidth = 597pt
				\paperheight = 845pt
			\end{verbatim} 
		\end{boxedminipage}
		\doublespacing


	% -------------------------------------- page -------------------
	\newpage  
	\subsection*{문서 여백 : geometry }

		\singlespacing
		\setlength{\fboxsep}{12pt}
		\begin{boxedminipage}[c]{1.0\linewidth}
			\begin{verbatim}
			\use package { geo metry }
			
			\geometry { paper size = { 25cm, 35cm }
			\geometry { total = { 20cm, 30cm }
			\geometry { body = { 18cm, 25cm }
			\geometry { hmargin = { 3cm, 2cm }
			\geometry { vmargin = { 2cm, 3cm }
			\geometry { margin par width = 2cm }
			\geometry { head = 1cm }
			\end{verbatim} 
		\end{boxedminipage}
		\doublespacing
	
		\singlespacing
		\setlength{\fboxsep}{12pt}
		\begin{boxedminipage}[c]{1.0\linewidth}
			\begin{verbatim}
			\use package 
			[ left=1.0in, right=1.0in, top=1.0in, bottom=1.0in ]
			{ geo metry }
			\end{verbatim} 
		\end{boxedminipage}
		\doublespacing












% ========================================== chapter ============================
\newpage
\chapter{머리말 꼬리말}



	% -------------------------------------- page -------------------
	\newpage
	\section{머리말 꼬리말}
	
% ========================================== chapter ============================
\newpage
\chapter{쪽 번호}
	
	% -------------------------------------- page -------------------
	\newpage
	\section{쪽 번호}
	
	
% ========================================== chapter ============================
\newpage
\chapter{쪽 나누기}
	
	
	% -------------------------------------- page -------------------
	\newpage
	\section{쪽 나누기}
	
		\begin{enumerate}[ topsep=0.0em, itemsep=-0.5em ]
		\item	\verb|\\| \\
				start a new paragraph.
		\item	\verb|\\*| \\
				 start a new line but not a new paragraph.
		\item	\verb|\- OK| \\
				to hyphenate a word here.
		\item	\verb|\cleardoublepage| \\
				flush all material and start a new page, start new odd numbered page.
		\item	\verb|\clearpage| plush all material and start a new page.
		\item	\verb|\hyphenation| enter a sequence pf exceptional hyphenations.
		\item	\verb|\linebreak| allow to break the line here.
		\item	\verb|\newline| request a new line.
		\item	\verb|\newpage| request a new page.
		\item	\verb|\nolinebreak| no line break should happen here.
		\item	\verb|\nopagebreak| no page break should happen here.
		\item	\verb|\pagebreak| encourage page break.
		\end{enumerate}
	
	



% ========================================== chapter ============================
\newpage
\chapter{줄 나누기}
		
	% -------------------------------------- page -------------------
	\newpage
	\section{줄 나누기}
	

% ========================================== chapter ============================
\newpage
\chapter{난 외주}

	% --------------------------------------- section -------------- 난 외주
	\newpage
	\section{난 외주}
	
		굉장히 쉽다\\
		\verb|\marginpar{text}|를 쓴다.\\
	
		\marginpar{난 외주 사용예}




























































\end{document}
