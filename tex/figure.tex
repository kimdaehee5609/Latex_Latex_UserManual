
\newpage
\chapter{그림}
 
% ------------------------------------------------------------- 그림 이란
	\newpage
	\section{그림이란 ?}
		그림(figure)이라고 부르는 것은 실제 그림뿐 아니라 텍스트의 일부, 표 등등 figure 환경으로 
		둘러싸인 것을 모두 가리킨다. 예를 하나 들면 그림 1에 보인 바와 같다.\\
	
		\begin{verbatim}
		\includegraphics[width=0.6\textwidth]{./fig/8.pdf}
		\end{verbatim}
		
		\includegraphics[width=0.6\textwidth]{./fig/8.pdf}
	

	\clearpage
% ------------------------------------------------------------- 그림 넣기 개요
	\section{그림 넣기 개요}
	
			그림을 삽입하려면 graphicx 꾸러미를 이용한다. 
				
			
			어떤 종류의 그림을 처리할 수 있는지는 전적으로 DVI(Device Independent) 드라이버에 달려 있다. 
			따라서 graphicx 꾸러미에 어떤 DVI 드라이버를 사용 하라고 다음과 같이 알려줘야 한다.
			\verb|\usepackage[dvips]{graphicx}|
			가장 널리 사용되는 DVI 드라이버는 dvips이다. 
			LATEX 대신 pdflatex을 쓰겠다면 pdftex이 필요하다.
			\verb|\usepackage[pdftex]{graphicx}|
			
			\paragraph{EPS}
			dvips는 EPS(Encapsulated PostScript)를 처리한다. 
			EPS는 점 방식(bitmap)일 수도 있고, 선 방식(vector)일 수도 있고, 그 두 가지가 혼용될 수도 있다. 
			그런 점에서 EPS는 가장 편리하고 신뢰할 만한 그림 형식이다.
			
			\paragraph{yap}
			yap은 EPS를 보여주거나 인쇄하지 못한다. 
			yap을 통해 DVI에 삽입된 EPS를 보려면 ghostscript가 필요하다. 
			
			고스트스크립트는 http://ftp.ktug.or.kr/mirrors/ghost/AFPL/에서 구할 수 있다.
			
			이제 그림을 삽입해보자.
			
			\verb|\includegraphcis{그림 파일}|
			
	\clearpage
	\section{includegraphics 명령의 지시어}
			
			다음은 \verb|\includegraphics|명령에서 가장 많이 쓰이는 지시자들이다.
			\begin{description}
			\item[scale]=숫자 \\
					그림 크기의 배율을 지시한다.
			\item[width]=길이 그림의 폭을 지시한다. \\
					폭만 정하면 높이도 그에 따라 바뀐다.
			\item[height]=길이 그림의 높이를 지시한다. \\
					높이만 정하면 폭도 그에 따라 바뀐다.
			\item[keepaspectratio]=true/false \\
					width와 height를 모두 지시했을 때, 폭과 높이의 비율이 바뀔 수 있다.
					이 지시자를 쓰면 지시된 폭과 높이를 넘지 않으면서 원래 비율을 유지시킨다.
			\item[angle]=숫자 \\
					시계 반대 방향으로 회전시킨다. angle의 값을 90으로 width 앞에 썼다면 width는
			height로 바뀐다.
			\end{description}
	
			\begin{description}
			\item[width=xx] \hfill \\
			Specify the preferred width of the imported image to xx.	 
			\item[height=xx] \hfill \\
			Specify the preferred height of the imported image to xx.
			\item[keepaspectratio] \hfill \\
			This can be set to either true or false. When true, 
			it will scale the image according to both height and width, but will not distort the image, 
			so that neither width nor height are exceeded.
			
			\item[scale=xx] \hfill \\
			Scales the image by the desired scale factor. e.g, 0.5 to reduce by half, or 2 to double.
			
			\item[angle=xx] \hfill \\
			This option can rotate the image by xx degrees (counter-clockwise)
			
			\item[trim=l b r t] \hfill \\
			This option will crop the imported image by l from the left, 
			b from the bottom, r from the right, and t from the top. Where l, b, r and t are lengths.
			
			\item[clip] \hfill  \\
			For the trim option to work, you must set clip=true.
			
			\item[page=x] \hfill \\
			If the image file is a pdf file with multiple pages, 
			this parameter allows you to use a different page than the first.
			\item[resolution=x]	\hfill  \\
			Specify image resolution in dpi
			\end{description}
	
	




% ------------------------------------------------------------- 그림의 위치
\newpage
\section{그림의 위치}

		그림들은 원고에 \textbf{지정된 정확한 그 위치에 나타난다는 것이 보장되지 않는다}는 사실에 주의하라. 
		사실 워드 프로세서와의 주요 차이점이 그림이 고정된 위치에 있는 것으로 취급되지 않는다는 점이다. 
	
		\paragraph{떠다니는 개체} 
			LATEX에서 그림은 LATEX 자신이 스스로 결정하는 적절한 위치로 “떠다니게” 되어 있다. 그러므로,
		본문에서는 “아래 그림”이나 “위의 그림”과 같이 참고하여서는 안되며, “그림 \ref{fig:label}을
		볼 것” 이런 식으로 써야 한다.
		이러한 속성 때문에 그림과 표는 떠다니는 개체(floats)라고 불린다. 
		
		\paragraph{그림의 위치를 특정하고 싶은 경우} 
		만약 이런 떠다니는 개체의 위치를 정확하게 특정하고 싶다면 here 패키지를 사용하여 위치 지정 옵션 인자를 H로 설정하면 된다.
	
	
% ------------------------------------------------------------- 그림의 배치
	\newpage
	\section{그림의 배치}
		
		\clearpage	
		\subsection{floatflt 꾸러미}
		
			\begin{verbatim}
			\begin{floatingfigure}[위치 지시자]{삽입할 위치의 폭}
			그림 삽입 명령어
			\caption{그림 제목}
			\end{floatingfigure}
			\end{verbatim}
		
			\begin{floatingfigure}[r]{0.33\textwidth}
			\includegraphics[width=.3\textwidth]{./fig/8.pdf}
			\caption{floatingfigure로 그림 배치하기}
			\end{floatingfigure}
			
			그림을 왼쪽 또는 오른쪽에 두고 그 옆으로 글이 흐르게 하려면 wrapfig 꾸러미를 이용한다. 
			위치 지시자에는 왼쪽 배치를 지시하는 $l$과 오른쪽 배치를 지시하는 $r$이 있다.
			
			floatflt와 비슷한 기능을 하는 꾸러미로는 wrapfig와 picins가 있다.
			floatingfigure 환경은 글 옆에 배치된다는 점 이외에 figure 환경과 다르지 않다.
			
			\begin{verbatim}
			\begin{floatingfigure}[r]{.33\textwidth}
			\includegraphics[width=.3\textwidth]{silver2}
			\caption{floatingfigure로 그림 배치하기}
			\end{floatingfigure}
			\end{verbatim}
		
		
		\clearpage
		\subsection{subfigure 꾸러미}
		
			한꺼번에 여러개의 그림을 넣고 각 그림마다 별도의 제목을 달려면 subfigure 꾸러미를 이용한다.
				
			\begin{verbatim}
			\begin{figure}
			\subfigure[그림 제목]{그림 삽입 명령어}
			\subfigure[그림 제목]{그림 삽입 명령어}
			\caption{그림 제목}
			\end{figure}
			\end{verbatim}
		
			\begin{figure}
			\centering
			\subfigure[8개월 때]{\includegraphics[width=.2\textwidth]{./fig/8.pdf}}\hfill
			\subfigure[16개월 때]{\includegraphics[width=.2\textwidth]{./fig/8.pdf}}\hfill
			\subfigure[32개월 때]{\includegraphics[width=.2\textwidth]{./fig/8.pdf}}
			\caption{subfigure로 그림 배치하기}
			\end{figure}
		
				
			\begin{verbatim}
			\begin{figure}
			\centering
			\subfigure[8개월 때]{\includegraphics[width=.3\textwidth]{silver}}\hfill
			\subfigure[16개월 때]{\includegraphics[width=.3\textwidth]{silver2}}\hfill
			\subfigure[32개월 때]{\includegraphics[width=.3\textwidth]{silver3}}
			\caption{subfigure로 그림 배치하기}
			\end{figure}
			\end{verbatim}
		
		
		
		\clearpage
		\subsection{한 페이지에 그림 하나만 중앙에 넣고 싶다}
		
			그림 전후에 \verb|\clearpage|를 넣어 주면 문제가 해결된다.
			\begin{verbatim}
			\clearpage
			\begin{figure}
			\centering
			\includegraphics{....}
			\end{figure}
			\clearpage
			\end{verbatim}
		
		
			\clearpage
			\begin{figure}
			\centering
			\includegraphics[width=1\textwidth]{./fig/8.pdf}
			\end{figure}
			\clearpage
		
		
			\clearpage
			\begin{figure}
			\centering
			\includegraphics[width=1\paperwidth]{./fig/8.pdf}
			\end{figure}
			\clearpage
		
			\clearpage
			\begin{figure}
			\centering
			\includegraphics[height=1\paperheight]{./fig/8.pdf}
			\end{figure}
			\clearpage
		
		
			\clearpage
			\begin{figure}
			\centering
			\includegraphics[page=1,scale=0.5]{./fig/8.pdf}
			\end{figure}
			\clearpage
	
% ------------------------------------------------------------- pdf page
\newpage
\section{Including full PDF pages}

	There is a great package for including full pages of PDF files: pdfpages. \par
	It is capable of inserting entire pages as is and more pages per one page in any layout (e.g. 2x3).

	\subsection{Package Options:}

	The package has several options:

	\verb|\usepackage[ options ]{pdfpages}|

	\begin{itemize}
	\item final : Inserts pages. This is the default.
	\item draf t: Does not insert pages, but prints a box and the filename instead.
	\item enable-survey : Activates survey functionalities. (Experimental, subject to change.)
	\end{itemize}
	
	The first command is

	\verb|\includepdf[ key=val ]{ filename }|
	
	Options for key=val (A comma separated list of options using the key = value syntax)
	
	\begin{description}
	\item[pages] 	Selects pages to insert. \\
			The argument is a comma separated list, \\
			containing page numbers (pages={3,5,6,8}), \\
			 ranges of page numbers (pages={4-9}) or any combination. \\
			To insert empty pages use {}. For instance pages={3,{},8-11,15} will insert page 3, an empty page, 
			and pages 8, 9, 10, 11, and 15.
			Actually not only links but all kinds of PDF annotations will get lost. 
			Page ranges are specified by the following syntax: m - n. 
			This selects all pages from m to n. Omitting m defaults to the first page; 
			omitting n defaults to the last page of the document. 
			Another way to select the last page of the document, is to use the keyword last. 
			(This is only permitted in a page range.) 
			E.g.: pages=- will insert all pages of the document, and pages=last-1 will insert all pages 
			in reverse order. (Default: pages=1) 
	\item[angle]	 You can use the angle-option for turning the included page, 
			for exampe for turning a landscape document when the latex-document is portrait. 
			Example: angle=90  
	\item[addtolist]	Adds an entry to the list of figures, 
			the list of tables, or any other list (e.g. from float.sty). 
			This option requires four arguments, separated by commas:
			addtolist={ page number , type , heading , label }
			\begin{itemize}
			\item page number : Page number of the inserted page.
			\item type: Name of a floating environment. (figure, table, etc.)
			\item heading: Title inserted into LoF, LoT, etc.
			\item label: Name of the label. This label can be referred to with \verb|\ref| and \verb|\pageref|.
			\end{itemize}
			Like addtotoc, addtolist accepts multiple sets of the above mentioned four arguments, all separated by commas. \\
			The proper recursive definition is: addtolist={ page number , type , heading , label [, lof-list ] }
	\item[pagecommand]  Declares LaTeX-commands, which are executed on each sheet of paper. 
			(Default : pagecommand=\verb|{\thispagestyle{empty}}| \\
			pagecommand=\verb|{\label{fig:mylabel}}|
	\end{description}
	
	You can also inserts pages of several external PDF documents. \par
	\verb|	\includepdfmerge[ key=val ]{ file-page-list } |
	Several PDFs can be placed table-like on one page. See more information in its documentation.		
			
				
	\includepdf[page=1]{./fig/8.pdf}
					
						
							
								
									
										
											
												
													
														
															
																
																	
																		
																			
																				
																					
																						
																							
																								
																									
																										
																											
																												
																													
																														
																															
																																
																																	
																																		
																																				

% ------------------------------------------------------------- 그림 넣기
\newpage
\section{그림의 넣기}

\subsection{latex로 처리}

	
	문서를 latex과 dvips로 처리하려 하는 경우, 그림은 EPS 파일만을 포함할 수 있다. 

\subsection{pdflatex로 처리}


		pdflatex은 PDF 포맷의 그림파일은 물론이고 JPG, PNG 그림도 처리한다. \\ \\
		만약 PDFLATEX을 사용한다면 .eps 그림을 반드시 .pdf로 바꾸고 원본에서도 .pdf 그림을 
		부르도록 수정해주어야 한다. EPS를 .pdf로 변환하는 데는 epstopdf를 이용한다.
		

		
						
		

% -------------------------------------------------------------
\newpage
\section{latex로 그림 넣기}

\rule{\linewidth}{1.2pt}
\begin{verbatim}
    \usepackage[dvips]{graphicx}
   
    \includegraphics[width=0.5\textwidth]{./fig/8}
\end{verbatim}
\rule{\linewidth}{1.2pt}


% -------------------------------------------------------------
\newpage
\section{pdf latex로 그림 넣기}

\rule{\linewidth}{1.2pt}
\begin{verbatim}
    \usepackage[pdftex]{graphicx}
    
    \begin{fingure}[htbp]
      \centring
      \includegraphics[angle=0,width=0.5\textwidth]{./fig/8.pdf}
      \caption{그림 설명문}
    	\end{fingure}
\end{verbatim}
\rule{\linewidth}{1.2pt}


\begin{table}[ht]
	\caption{graphicx 패키지의 key 명령}
	\centering
    \begin{tabular}{l l}
    \toprule
    width   & 그림의 폭을 지정   \\
    height  & 그림의 높이를 지정   \\
    angle   & 그림을 반시계 방향으로 회전   \\
    scale   & 그림의 배율을 지정  \\
    \bottomrule
    \end{tabular}%
  \label{table:keyname}%
\end{table}%




% -------------------------------------------------------------
	\newpage
	\section{컴파일에 따른 그림 파일의 선택적 사용}

		.pdf와 .ps를 같은 원본 파일에서 컴파일러에 따라 선택적으로 사용하도록 하려면 다음과 같이 하
		라.
		
		\rule{\linewidth}{1.2pt}
		\begin{verbatim}
			% define the variable \ifpdf
			\newif\ifpdf
			\ifx\pdfoutput\undefined
			\pdffalse
			\else
			\pdfoutput=1
			\pdftrue
			\fi
			...
			% include the right options
			\ifpdf
			\usepackage[pdftex]{graphicx}
			\pdfcompresslevel=9
			\else
			\usepackage{graphicx}
			\fi
			...
			% include the right graphic file
			\ifpdf
			\includegraphics{file.pdf}
			\else
			\includegraphics{file.eps}
			\fi
		\end{verbatim}
		\rule{\linewidth}{1.2pt}

% -------------------------------------------------------------
%
%	에러 시 대처
%
% -------------------------------------------------------------
	\newpage
	\section{에러시 대처}

		18개 이상의 figure(떠다니는 개체)들이 대기열에 들어가서 처리되지 않고 있으면 ‘Too many
		unprocessed floats’라는 LATEX 에러를 만나게 된다. 이럴 경우 가장 간단한 방법은 nclearpage
		명령을 세 개 또는 네 개의 그림 뒤에 붙여서 floats들을 처리하고 페이지를 나누게 하는 것이다.
		참고로 more°oats 패키지를 쓰면 대기열에 저장할 수 있는 floats의 개수가 36개로 늘어난다.

% -------------------------------------------------------------
%
%	그림의 형식
%
% -------------------------------------------------------------
\newpage
\section{그림의 형식}

		\subsection{그림의 형식}
			
				그림이 Encapsulated POSTSCRIPT (.eps) 포맷이라면, graphicx 패키지를 써서 LATEX 원본 파일에
				삽입할 수 있다. 그림 2에 보인 것과 같은 명령을 사용할 수 있다.
		
		
		\subsection{그림의 형식 변환}
		
				.jpg, .gif, .png와 같은 일반적인 그래픽 포맷을 .eps로 변환시켜주는 프로그램이 많다. 그 중에
				예를 들면 ImageMagick (http://www.imagemagick.org)과 GIMP (http://www.gimp.org)를 들 수
				있다. 다만, 이 프로그램들이 만들어내는 POSTSCRIPT 파일은 크기가 엄청나다.
				
				비트맵 그림을 작은 크기의 POSTSCRIPT 파일로 변환해주는 방법으로는 jpeg2ps (http://www.
				pdflib.com/jpeg2ps/index.html)나 bmeps (CTAN://support/bmeps)가 좋다. 
				앞의 것은 .jpg 파일을 변환할 때는 가장 좋은 방법이고, 나중의 것은 더 많은 포맷의 그림파일을 다룰 수 있다.
		
