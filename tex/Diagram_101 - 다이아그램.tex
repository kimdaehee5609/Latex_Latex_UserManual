	
% ========================================== chapter ============
\newpage
\chapter{Diagram}


		% ----------------------- 장 목차 생성 Initialization
		\minitoc		


% -------------------------------------------------------------
	\newpage
	\section{Diagram 개요}
	
	\setlength{\fboxsep}{12pt}
	\begin{boxedminipage}[c]{1.0\linewidth}
	\linespread{1.0}
	\small
	\begin{verbatim}
		\begin{tikzpicture}[
		roundnode/.style={	circle, 
							draw=green!60, 
							fill=green!5, 
							very thick, 
							minimum size=7mm},
		squarednode/.style={	rectangle, 
							draw=red!60, 
							fill=red!5, 
							very thick, 
							minimum size=5mm},
		]
		%Nodes
		\node[squarednode]	(maintopic)		{2};
		\node[roundnode]     (uppercircle)		[above=of maintopic] {1};
		\node[squarednode]	(rightsquare)	[right=of maintopic] {3};
		\node[roundnode]	(lowercircle)	[below=of maintopic] {4};
 		%Lines
		\draw[->] (uppercircle.south) -- (maintopic.north);
		\draw[->] (maintopic.east) -- (rightsquare.west);
		\draw[->] (rightsquare.south) .. controls 
					+(down:7mm) and +(right:7mm) .. (lowercircle.east);
	\end{verbatim}
	\normalsize
	\end{boxedminipage}

		\begin{tikzpicture}[
		roundnode/.style={	circle, 
							draw=green!60, 
							fill=green!5, 
							very thick, 
							minimum size=7mm},
		squarednode/.style={	rectangle, 
							draw=red!60, 
							fill=red!5, 
							very thick, 
							minimum size=5mm},
		]
		%Nodes
		\node[squarednode]	(maintopic)		{2};
		\node[roundnode]     (uppercircle)		[above=of maintopic] {1};
		\node[squarednode]	(rightsquare)	[right=of maintopic] {3};
		\node[roundnode]	(lowercircle)	[below=of maintopic] {4};
 		%Lines
		\draw[->] (uppercircle.south) -- (maintopic.north);
		\draw[->] (maintopic.east) -- (rightsquare.west);
		\draw[->] (rightsquare.south) .. controls 
					+(down:7mm) and +(right:7mm) .. (lowercircle.east);

		\end{tikzpicture}
		
		\paragraph{ $\backslash$node   [ squarednode ] (maintopic) \{ 2 \} ; }

			\begin{itemize}
			\setlength\itemsep{0.0em}
			\item 이전 명령에 정의 된대로 squarednode를 작성합니다. 
			\item 이 노드는 아이디를 maintopic로 하고, 숫자 2를 포함합니다. 
			\item 괄호 안에 빈 공간을 둘 경우 텍스트가 표시되지 않습니다. 
			 \end{itemize}
			 
		\paragraph{[above=of maintopic]}

			추가 매개 변수가 
			첫 번째 노드를 제외한 모든,
			이 매개 변수가 다른 노드에 상대적인 위치를 결정하는 것을 알 수 있습니다. 
			[above= maintopic] \\
			
			예를 들어 이 노드는 노드 이름 maintopic 위에 표시해야 함을 의미한다. 
			이 위치 확인 시스템이 작동하려면 프리앰블에 $\backslash$usetikzlibrary\{positioning\}를 
			추가해야합니다.



% -------------------------------------------------------------
	\newpage
	\section{Diagram 블럭 그리거}

		% -------------------------------------------------------------------    
			\begin{tikzpicture}
		    		[node distance=2cm,>=latex',
				 block/.style = {draw, 	shape=rectangle, align=center} ]
			\node [block]                      		(start)     	{ 일반 블럭};
		    	\end{tikzpicture}
		    	
		% -------------------------------------------------------------------    
		    \begin{tikzpicture}
		    		[node distance=2cm,>=latex',
				rblock/.style = {draw, 	shape=rectangle, rounded corners=0.5em} ]
			\linespread{1.0}
			\node [rblock]                      		(start)     	{ r 블럭 (rectangle)};
		    	\end{tikzpicture}
		    	
		% -------------------------------------------------------------------    
		    \begin{tikzpicture}
		    		[node distance=2cm,>=latex',
				tblock/.style = {draw, 	shape=trapezium, 	
										trapezium left angle=60, 
										trapezium right angle=120, 
										align=center} ]
			\node [tblock]                      		(start)     	{ t 블럭 (trapezium)};
		    	\end{tikzpicture}


% -------------------------------------------------------------
	\newpage
	\section{블럭 배치}




% -------------------------------------------------------------
	\newpage
	\section{화살표 배치}

		    			    	

% -------------------------------------------------------------
	\newpage
	\section{Diagram 그리기}



	% ---------------------------------------------- page 
	\subsection{다이아그램 그리기 일반}
	\null

		% -------------------------------------------------------------------    
		%  다이아그램 그리기
		% -------------------------------------------------------------------    
		    \begin{tikzpicture}
		    		[node distance=2cm,>=latex',
				 block/.style = {draw, 	shape=rectangle, align=center},
				rblock/.style = {draw, 	shape=rectangle, rounded corners=0.5em},
				tblock/.style = {draw, 	shape=trapezium, 	
										trapezium left angle=60, 
										trapezium right angle=120, 
										align=center},
		             ]
		             
			\linespread{1.0}
			\node [rblock]                      		(start)     	{출발\\ Start};
		    	\node [block, right=of start]       	(acquire)   	{Node2 \\ Acquire Image};
		    	\node [block, right=of acquire]     	(rgb2gray)  	{RGB to Gray};
		    	\node [block, right=of rgb2gray]   	(otsu)      	{Localized OTSU\\ Thresholding};
		    	\node [block, below=of acquire]   	(gchannel)  	{Subtract the\\ Green Channel};
		    	\node [block, right=of gchannel]   (closing)   	{Morphological\\ Closing};
		    	\node [block, right=of closing]     	(detecting) 	{Sign Detectiing\\ using NN};
		    	\node [tblock, right=of detecting] (speed)     	{Speed\\ Limit};
		    	
			\draw[->] (start) -- (acquire);
			\draw[->] (acquire) -- (rgb2gray);
			
		%etc
			\coordinate[below right=5mm and 5mm of otsu]   (a1);
			\coordinate[left=5mm of a1 -| gchannel.west]   (a2);
			\draw[->] (otsu) -| (a1) -- (a2) |- (gchannel);
		%etc
		    	\end{tikzpicture}
		    	
		    	
		    	
		    	
	% ---------------------------------------------- page 
	\newpage
	\subsection{아래방향으로 다이아그램 그리기}
	\null
    	
		% -------------------------------------------------------------------    
		%  아래 방향으로의 다이아그램 그리기
		% -------------------------------------------------------------------    
		
		    \begin{tikzpicture}
		    		[node distance=1cm,>=latex',
				 block/.style = {draw, 	shape=rectangle, align=center},
				rblock/.style = {draw, 	shape=rectangle, rounded corners=0.5em},
				tblock/.style = {draw, 	shape=trapezium, 
										trapezium left angle=60, 
										trapezium right angle=120, 
										align=center},
		             ]
		             
			\linespread{1.0}
			\node [rblock]                      		(start)     	{출발\\ Start};
		    	\node [block, below=of start]       	(acquire)   	{Node2 \\ Acquire Image};
		    	\node [block, below=of acquire]   	(rgb2gray)  	{RGB to Gray};
		    	\node [block, below=of rgb2gray]  (otsu)      	{Localized OTSU\\ Thresholding};
		    	\node [block, below=of otsu]   	(gchannel) 	{Subtract the\\ Green Channel};
		    	\node [block, below=of gchannel]  (closing)   	{Morphological\\ Closing};
		    	\node [block, below=of closing]    (detecting) 	{Sign Detectiing\\ using NN};
		    	\node [tblock, below=of detecting] (speed)     	{Speed\\ Limit};
		    	
			\draw[->] (start) -- (acquire);
			\draw[->] (acquire) -- (rgb2gray);
			\draw[->] (rgb2gray) -- (otsu);
			\draw[->] (otsu) -- (gchannel);
			\draw[->] (gchannel) -- (closing);
			\draw[->] (closing) -- (detecting);
			\draw[->] (detecting) -- (speed);
			
			
    	\end{tikzpicture}
    




















